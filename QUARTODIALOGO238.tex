\chapterspecial{Quarto diálogo -- Eu sou um negro}{}{}
 

Neste quarto diálogo, Jacob Pinheiro Goldberg começa falando do medo da
morte. Somos regidos pela pulsão de vida e pela pulsão de morte. Segundo
ele, esta dualidade é espelhada na cultura através de vários dualismos,
como o bem e o mal, ou Deus e Satã.

Inserido na cultura, não vemos mais o ser humano apenas, mas encontramos
um contexto cultural que pode ser analisado pela Psicanálise Social. No
caso brasileiro, há uma dualidade percebida como metanoia e paranoia,
que rege o comportamento social. Metanoia significa a transformação~ da
forma de pensar, a busca do caminho certo através da transformação do
ser, da celebração da vida, enquanto que paranoia significa um estado de
desconfiança patológica, erros de interpretação da realidade, e de uma
forma mais popular, loucura ou maluquice.

Esta situação cultural faz com as pessoas vivam no Brasil num constante
movimento de ação e reação, metanoia e paranoia. Quando a pessoa sente
medo, ela supera esse medo através de um exercício de ataque, de
ofensiva. Se ela sabe que vai morrer, tenta viver intensamente. Essa
pessoa muitas vezes procura tratamento imaginando que está doente,
quando na verdade está tendo é uma manifestação de sanidade.

Na medida em que somos todos brasileiros, há menos atrito e repulsa
entre os descendentes de português, espanhol, italiano, coreano,
japonês, judeu, e mais recentemente até os índios são trazidos para esse
processo de paranoia, sob a égide da ideia de ``ordem e progresso''. Mas
de alguma maneira, isto é contraposto pela metanoia, que é a alegria de
viver emanada pelos negros e pardos com samba, com carnaval, e todo
mundo adere. Mas eles continuam a ser os grandes excluídos da festa.

Segundo Goldberg, não existe vontade política, a questão é psicológica.
A~oligarquia no Brasil é sádica e cruel, e o povo é masoquista. A~oligarquia não tem pátria. Grande parte dessas famílias oligarcas sempre
tiveram um pé no Brasil e um pé na Europa. Os filhos sempre foram
educados para pensar como europeus. É~como se os filhos dos antigos
colonizadores tentassem manter este movimento de subjugar os negros e
pardos, e até pouco tempo atrás os índios, dentro do país.

O Estado brasileiro passa o tempo todo tentando escravizar as pessoas em
nome da liberdade. Isso é paranoia. A~resistência é exercida através da
malandragem. Quem utiliza esta malandragem são os negros, através da
postura de se fazer de besta, fingir que nada está acontecendo, que é o
limite da resistência.

Por outro lado, verificamos que a criança e o adolescente acabam tendo o
mesmo papel dentro da família. Estes são os sintomas de bode expiatório
dessa realidade paranoica. Nesse sentido, a psicanálise é o vestibular
de evolução.

Infelizmente, muitas vezes a psicanálise é mal entendida, e serve de
sistema de frenagem. A~pessoa está num momento com a intenção de viver,
de reagir contra tudo aquilo que o aleija espiritualmente. Não pode
perder tempo conversando sobre o que acontecia quando tinha oito anos de
idade.

\begin{center}\asterisc{}\ ~\end{center}

\abrefala

\textbf{Jacob Pinheiro Goldberg}: De certa forma a informação da morte
é tratada de maneira genial por Schopenhauer; que alguns acham que é
unicamente cética, unicamente pessimista e de caráter negativo. Não é
verdade. Existe um transbordo na ideia de Schopenhauer, que flerta
claramente com o conceito de eternidade. Ela não se esgota
absolutamente, numa constatação de finitude.

 

Inclusive, a associação que ele faz entre o sono, o sonho, a morte,
provavelmente pode ter sido uma das inspirações de Jorge Luis Borges. A~respeito da questão não só da morte, do sono, e do sonho, mas também a
questão do especular, que fascina o tempo todo o Borges. Mas é também o
tempo todo um diálogo em Schopenhauer. Ele sai do monólogo, quando
dialoga com ele mesmo de maneira especular.

 

Ontem, relendo um texto de Schopenhauer, eu reví, dentro dos escaninhos
da minha memória, essa ideia. A~neurose é o poema do inconsciente. E~a
neurose é a formulação de um poetastro. É~um poeta menor, que deitado no
divã, fica apavorado diante de um susto permanente que ele tem, diante
dos seus desacertos. Ele sabe que dentro dele coabitam, e entram em
conflito permanentemente, duas grandes forças com as quais ele não sabe
lidar.

 

Mas ele tem a pré-ciência de que são elas que ordenam ou promovem a
desordem de sua vida, que o leva ao sofrimento da neurose. Essas forças
são a pulsão de vida e a pulsão de morte. A~crueldade e a beatitude. O~bem e o mal. Deus e Satã. Quando Deus e o diabo estão na Terra do Sol,
os miolos esquentam. Quando os miolos esquentam, um povo vive
permanentemente em estado de \versal{PMD} (Psicose Maníaca"-Depressiva). Ou o
Brasil é o maior país do mundo, ou nós estamos na beira do abismo.

 

Quando acontece isso, não são os ``Tristes Trópicos'' de Claude
Levi"-Strauss. É~um vulcão em pré ebulição, e às vezes, quando esse vulcão
solta lavas, como foi no caso da ditadura iniciada por um psicopata
paranoico que foi Mourão Filho, que saiu de Juiz de Fora para fazer o
que fez, para cometer o que cometeu. Nada disso é por acaso, são as
lavas do vulcão. Esse vulcão não é só Brasil é a América. Da qual extrai
meu livro ``O Feitiço da Amérika''.

 

Eu desenhei uma ideia desse esquema, que na verdade forja um molde de
psicologia do brasileiro, completamente diferente, eu diria, quase
oposta àquela concebida por Gilberto Freire ou Darci Ribeiro. Pensei
um caboclo confuso, mais para dementado do que para realmente um
pensador equilibrado, tentando criar uma concepção de Brasil. Isto me
custou o que me custou, em termos de polêmica.

 

\textbf{Renato Bulcão}: Sim, mas na sua opinião isso significa que, seja
pelo viés do caboclismo darwiniano, ou no viés bastante discutido de
``Casa Grande e Senzala'', isso é pouco para explicar a brasilidade.

 

Porque dentro da sua mineirice, dentro da sua origem judaica, dentro da
sua formação protestante, você é mais brasileiro do que a maioria dos
brasileiros. A~sua visão tem que estar construída em cima de uma nova
estrutura, diferentemente dessas pessoas, que de alguma maneira se acham
descendentes seja de índios, seja de negros, seja de portugueses; porque
claramente você não é descendente de nenhum desses.

 

Então, você tem a facilidade de poder ter um outro olhar. Ter um novo
olhar do Brasil, que é um país de migração, que é outro Brasil, mas que é
o mesmo Brasil.

 

\textbf{G}: É impressionante, porque eu acho que você recolheu todos os
elementos que são característicos da posição na qual eu me situo dentro
do pensamento e dentro da ação na cena brasileira. Você colocou
exatamente como eu vejo.

 

É interessante que de alguma forma esta sua visão coincide com as teses
que têm sido feitas de análise do que eu venho arquitetando.

 

Eu falo do negro sem ser negro. Eu falo do índio sem ser índio, eu falo
das raízes brasileiras sem ser filho de português.

 

Nós sabemos, inclusive, que nos primórdios, a raiz brasileira não tem
nada a ver com Portugal. Muito pelo contrário, Portugal é o mais
estrangeiro dos grupos que aqui aportaram. O~conflito entre Portugal e
Brasil, a história diz isso, foi violentíssimo. Até hoje, inclusive, o
ódio contra o português habita todas as anedotas e piadas; o ridículo e
o grotesco e vice"-versa. Não obstante todos os esforços que têm sido
feitos em cima do idioma comum. Não por acaso, até um determinado
instante, eu acredito que eu tenha sido aquele que emprestou à ideologia
nacionalista brasileira, a única visão de uma esquerda, independente de
qualquer peso stalinista.

 

O jornalista Osvaldo Costa, diretor de ``O Sumário'', saiu do Rio de
Janeiro, encontrou"-se comigo numa pensão pobre, porque naquela época
jornalista era pobre. Hoje a gente sabe que jornalista é um marqueteiro
a serviço dos grandes interesses. Naquela época, jornalista mal tinha o
que comer. Isso acontecia também com Samuel Wainer no fim da vida.
Quando eu fui falar com o vereador Davi Roisen para arrumar um
dinheirinho na maçonaria, para ele poder se sustentar e continuar o
jornal dele aqui em São Paulo, depois de ter sido acusado por
Carlos Lacerda de ter roubado o Banco do Brasil e ter ficado milionário.
Se roubou, não sei para onde é que foi o dinheiro. Quando eu o
encontrei, ele tinha dificuldade de sobreviver financeiramente. Aliás,
nunca recebi um tostão para escrever no seu jornal.

 

Naquela época, quando eu pensei no Brasil, e percebi algumas vertentes
desse nacionalismo autêntico ao qual você acaba de se referir, e que não
tem nada de chauvinista, muito pelo contrário. Ele é ligado ao que
existe de melhor no cosmopolitismo, no respeito das diferenças, seja de
negro, de índio ou de americano.

 

Na minha formação tem o missionarismo, o protestante norte"-americano. A~música de fundo da minha biografia é o country americano. Quando fecho
os olhos, eu me imagino um cowboy no \emph{saloon}. Eu não me imagino um quixote
espanhol, eu me imagino um Zorro nas causas justas. É~dessa maneira que
eu tenho vivido a minha vida. É~isso que justifica as minhas ideias.

 

Basicamente, a forma da psicanálise que eu enxergo, se for reduzir ao
máximo, o sujeito só existe no anseio da integralidade, no anseio da
completude da maximização dos seus recursos enquanto pessoa, ele só
existe quando se opõe ao coletivo, ao social, o que fundamentou minha
tese de doutoramento no Mackenzie.

 

Não por acaso com 80 anos de idade, eu subo ao palco do teatro com o
cômico Eduardo Sterblitch e a gente cria o Teatro da Psicanálise Social.
Eu vou para o palco de stand"-up, na frente de 400 pessoas, desmistificando
de uma vez a ideia do consultório aristocrático sagrado, que tem tudo a
ver com as neves da Europa, com o gelo da Europa. Lá você tem que deitar
no divã, com o cobertor em cima, porque está frio. Aqui nós estamos em
combustão. Aqui, se você deixar, o paciente começa a cantar e dançar
samba na tua frente.

 

Qual é a interlocução da psicanálise? Aqui dentro é metanoia e paranoia.
Essas são as duas grandes vertentes, na minha opinião, do contributo que
eu tenho procurado prestar ao entendimento da psicanálise. Não são as
aflições gregas de Freud com o conceito da mãe judia que ele transporta
para o Édipo grego. Aqui tem a babá negra, a Maria que me contava a
história da mula"-sem"-cabeça (Palestra que apresentei na Universidade de
Stanford).

 

\textbf{R}: Na sua opinião, em Freud há uma transposição da cultura
judaica, transformada ou fantasiada em cultura grega, para um
determinado ajuste social? Na sua opinião, é isso que acontece?

 

G: Não tenha dúvida. Naquele momento, ainda mergulhado no antissemitismo
do Império Austro"-Húngaro, do antissemitismo europeu, do antissemitismo
alemão, seria impossível para Freud, como ele deixa claro mil vezes, que
se ele amparasse as suas ideias em cima de uma mística judaica, a oposição seria muito maior do que foi. Então ele joga, na minha opinião,
com os elementos de helenização, para dessa forma facilitar a
compreensão, a aceitação.

 

Da mesma forma que Freud concordou em assimilar a cultura epidérmica de
Jung, posteriormente inclusive, que tem uma identificação profunda com o
hitlerismo, porque ela não é superficial. Ficam aí discutindo se Jung
foi membro do partido (nazista). Claro que não foi membro do partido.
Não foi lá buscar carteirinha. Mas o pensamento dele, não tenha dúvida
nenhuma, é o pensamento dos sinais, das mágicas, dos fetiches, das
crendices e supertições. É~o homem primitivo, é o homem da caverna,
transvestido de mensagens extraordinárias do Oriente.

 

Um Oriente muito mal elaborado, muito mal entendido. Mal compreendido,
inclusive, para justificar uma alma conturbada, como ele mesmo
reconhece, em ``Memórias, Sonhos e Reflexões'', em que ele mesmo se
confessa ter vivido um processo de psicose durante não sei quantos anos.
Posteriormente ficam as tietes acreditando que aquilo foi um romance de
formação. Não foi um romance de formação coisa nenhuma. Foi doença, o
nome daquilo é doença. Ele tinha se desequilibrado mentalmente. Claro
que ele soube elaborar tudo isso, elaborou razoavelmente até. Arendt e
Heidegger: a vítima internaliza o fetiche conosco.

 

Mas Freud precisava dele. Como ele, seria o nosso príncipe loiro. Eu não
preciso de príncipe loiro nenhum. Eu bato de frente com Gilberto Freire,
eu bato de frente com Darci Ribeiro. Eu digo que eu acho 
que eles, independente de patrulha
ideológica, independente da idade das trevas que nós vivemos na ditadura
são cúmplices do erro.

 

Eu fui fazer um trabalho de análise sobre o desempenho do Brasil na Copa
(do Mundo de Futebol) em Paris. Quando eu voltei, o Boris Casoy fez uma
entrevista comigo e perguntou ``Professor, por que o Brasil perdeu o
jogo?''. Eu disse: ``Porque os franceses têm a Marselhesa''. Quando eu vi
o presidente da França pondo a mão no peito e começar a cantar a
Marselhesa, eu virei para uma pessoa que estava comigo e falei: ``Vamos
embora, eu não vou assistir a derrota.'' Diante desse hino cantado dessa
maneira, eu me lembro de Napoleão: me dê cinco mil homens e a
Marselhesa cantada com fervor que eu venço dez mil homens.

 

Eu fui dar aula na \versal{USP}, na Faculdade de Música, a respeito da importância
do som na formação psíquica dos povos. O~Gilberto Vasconcelos elogiou o
meu ``Psicologia da Agressividade'' por este enfoque sonoro"-teórico, na
Folha de São Paulo (jornal da cidade de São Paulo).

 

Na minha obra, é mais ou menos dessa maneira que eu sempre tenho
enxergado a psicanálise: ação e reação, metanoia e paranoia. Se estou
com medo, eu ultrapasso esse medo através de um exercício de ataque, de
ofensiva. Se eu sei que vou morrer, eu tento viver intensamente. É~um
convite que eu faço para o sujeito que me procura e imagina que ele está
doente, quando na verdade está tendo é um um assomo de sanidade.

 

O sujeito, quando bate o pé e procura a psicanálise, é o vestibular de
evolução. Infelizmente muitas vezes, a psicanálise mal entendida serve
de sistema de frenagem. O~sujeito vai lá porque ele está no momento de
intenção de viver, de reagir contra tudo aquilo que o invadiu, que o
aleija.

 

Nesse instante, em vez do psicanalista ser o preparador que vai para o
campo de esporte, ele permite que o paciente fique lacrimejando em cima
da experiência de fazer pipi nas calças durante um ano.

 

\textbf{R}: Três questões surgem então. Primeira questão: metanoia e
paranoia passam a ser a motivação, os princípios que causam as reações
mais visíveis do povo brasileiro de alguma forma.

Segunda questão: no momento em que o sofá está no palco e há um convite
para que a plateia eventualmente venha ao sofá, existe alguma vontade de
interpretação disso, ou a gente não deve ficar interpretando a vida
comum como muitas vezes propõe se propõe a psicanálise tradicional.

A última questão: na medida em que somos todos brasileiros, todo mundo
convive bem com coreano, japonês, judeu, espanhol, português, o índio,
inclusive, são trazidos a essa metanoia. Mas, de alguma maneira, quem
fabrica a metanoia de verdade são os negros, com samba, com carnaval,
com as maiores manifestações de alegria e que todo mundo adere, que são,
na verdade, os grandes excluídos dessa festa.

 

\textbf{G}: Eu acho que como alguém que generosamente tem lido,
estudado e compartilhado do meu trabalho durante alguns anos, que é o
teu caso, você captura muito bem, e às vezes até melhor do que eu,
determinados signos do meu trabalho. Eu nunca tinha pensado nisso. Em um
livro meu ``Judaísmos: Ético e não Étnico'' eu cito o Gilberto Gil.
Aliás, o dia que eu contei isso pra ele, ele ficou muito comovido: ``Bob
Marley morreu, porque além de negro, era judeu'', é o verso do Gilberto
Gil.

 

Quando você estava falando, ficou claro para mim, algo que provavelmente
eu vou passar a usar. Eu sou um frasista viciado. Se alguém me perguntar
aquela pergunta tão agressiva, quanto estúpida: ``O senhor é mesmo?'',
eu vou ter que responder: um negro!

Porque talvez agora ficou mais claro para mim mesmo qual é a minha
condição. É~de negro! Porque se o negro é o grande elemento excluído da
sociedade, mas na verdade ele é o que dá a pirueta, a volta por cima.
Porque é ele quem dá a volta por cima. Ele que faz a diabolização
através do carnaval, através do samba e através, por que não… da
porrada? E~chegando à fronteira do crime, por que não? Essa a minha
identificação. Essa é minha posição filosófica dentro da própria
psicanálise.

A psicanálise é um convite para a negritude. Não por acaso, o lugar onde
talvez eu tenha me sentido mais eu em toda a minha vida foi no Senegal.
Foi exatamente na última viagem do Concorde Rio"-Paris, o voo histórico
para Copa.

Quando eu desci no aeroporto, eu disse ``vou dar uma volta porque eu tenho
que dar um pulo até a Universidade''. O~policial negrão vira pra mim e
fala ``E o passaporte? O senhor tem o visto do Senegal?''. Eu jamais
imaginei que precisaria. Ele falou ``Então o senhor não vai sair do
aeroporto''. Eu olho pra ele e me lembro do poema do Leopold Sendar
Senghor, o primeiro presidente do Senegal , grande poeta da negritude.
Olho para ele e declamo. Quando acabo de declamar, os olhos dele estão
marejados de lágrimas. E~ele me convida para visitar a universidade. Foi
realmente um lugar em que eu me senti dentro, \emph{inside}. Por que foi
no Senegal?

Por duas razões; primeiro, porque quando meu pai pegou um navio
argentino chamado ``Valdivia'', e meu pai semi clandestino, embarcadiço
sem dinheiro, com 18 anos de idade disse: ``Eu quero ir para o Rio de
Janeiro, porque me disseram que na Argentina dá pra arrumar emprego''. E
ele veio foi por engano. O~navio parou no Senegal. Meu pai contou que o
dia que ele viu o primeiro negro ele quase desmaiou. Porque na Europa,
um sujeito como ele do interior da Polônia, não sabia que existiam
negros. E~ele me contou que ele ficou encantado de ter visto uma pessoa
negra.

Eu tive a mesma sensação dele olhando aqueles negros. Porque era um
negro que tinha um olhar orgulhoso. Era um negro que tinha o olhar de
homem dono do seu destino. Diferente do negro brasileiro, massacrado, e
que só através da oposição, da contestação, da resposta, da recusa, da
revolta, do carnaval, da esculhambação é que pode se afirmar enquanto
pessoa.

É dessa maneira que eu enxergo a psicanálise, é só dessa maneira. Quando
a família vem procurar a psicanálise e traz a criança para ser tratada,
não tenha dúvida que a criança é porta"-voz da neurose do grupo. O~adolescente idem, ele usa a droga por quê?

Eu fiz aquele trabalho que acabou redundando no livro ``Geração
Abandonada'', que ganhou prêmio de jornalismo do Rei da Espanha. Os reis
têm muita competência para cooptar. No fim foram oito dias, duas páginas
inteiras cada dia, do jornal Estado de São Paulo. Uma pesquisa que mexeu
com a psicologia do Brasil refletida no ``Fantástico'', \versal{TV} Globo, \versal{SBPC},
e o \versal{DOPS} me perseguindo. E, depois, o silêncio tumular que tudo
cobre…

 

Aí você me pergunta ``Por que você chegou até aí?''. É evidente e óbvio
que era minha intenção revoltar, promover uma revolução. Depois o
``Estadão'' (jornal Estado de São Paulo) deu mais uma página inteira
sobre o que eu considerava psicanálise. Eu usei o recurso do
``Estadão'', do ``Jogo da Verdade'' na \versal{TV} Cultura, sempre com esse
objetivo. É~curioso porque aqueles de malícia, de má fé, sempre
perceberam isso no meu comportamento. Eu sempre tentei obviamente
disfarçar, porque de tonto eu não tenho nada, de malandro do bem eu
tenho tudo. A~malandragem me foi ensinada pelo antissemitismo, pela
brutalidade, pela violência, pelo bullying na infância, pela exclusão.
Ou eu era malandro do bem, ou não sobrevivia.

 

Aliás, foi esta grande parte da resposta da sociedade brasileira diante
do nazismo, e da ditadura. A~sociedade brasileira em grande parte se
fingiu de besta, que era a única forma de resistir. Porque o que a
ditadura queria era um massacre, diante do qual o golpe no Chile teria
sido um ensaio. Porque gana eles tinham. O~sadismo existia.

 

Escrevi meu artigo analisando João Goulart. Então, promoviam aqui delação
e provocação permanente. Você não podia confiar em ninguém. O~sujeito
entrava dentro da sua casa, alegadamente com carteirinha de esquerda, e
você não sabia se o sujeito não era um agente do \versal{SNI} ou do \versal{DOPS}. Você
tinha que tomar cuidado. A~gente está vendo agora que realmente muita
gente que se mascarava de esquerda e assim por diante, fazia o trabalho
do cabo Anselmo. Isso foi um dos estragos na alma desse psiquismo, da
paranoia, que até hoje a gente vive de alguma maneira.

 

A gente só tem uma forma de reagir a isso. Semana passada eu falei na
plenária do \versal{PSOL} (partido do qual eu me desliguei, aliás), na mesma linha
que eu fiz um discurso no teatro do ensaio geral lá com o Eduardo
Sterblitch. Quando um deputado que é um ícone hoje da direita do Brasil,
que se pretende a própria figuração da virilidade, ficou fazendo
críticas contra os comunistas, os homossexuais e assim por diante, e
depois quis tomar atitude da casa"-grande, perguntando para o Eduardo:
``Você aceita um abraço?'' e foi abraçar o Eduardo.

 

O Eduardo o abraçou diante da câmera e de milhões de pessoas, e
desmoralizou de vez a hipocrisia da direita brasileira, que é covarde, e
passou a mão na bunda do deputado, e o deputado saiu sem graça.

 

Como a Ação Integralista Brasileira saiu pelas portas dos fundos quando
foi desmascarada em seu facismo genético do \versal{SIGMA}.

 

\textbf{R}: Voltando aqui a paranoia e metanoia: nós temos, então,
especificamente, um país que tem a sua população tentando conviver em
algum tipo de termo, mas estamos claramente prejudicando uma parte
étnica desse povo.

 

Nós temos a posição da paranoia, que seria a posição da ordem e
progresso, que seria a posição de alguma maneira atribuída ao pensamento
kantiano. Há, de alguma maneira, a ideia da integridade, da honra e das
virtudes. Isso cria um estado de paranoia, é um estado que acaba
fingindo que é estado, um estado que acaba impedindo os direitos das
pessoas.

 

\textbf{G}: É um estado que passa o tempo todo tentando escravizar as
pessoas em nome da liberdade.

 

\textbf{R}: Por outro lado, nós temos o que o senhor chamou de
malandragem…

 

\textbf{G}: Que é na verdade a resistência.

 

\textbf{R}: Também colocou como malandragem a ideia de se fazer de
besta, que seria a possibilidade da resistência, e ao mesmo tempo
colocou que a criança e o adolescente acabam tendo um mesmo papel dentro
da família.

 

\textbf{G}: Eles fazem o papel do negro.

 

\textbf{R}: O papel do negro instituído dentro da família. A~criança e o
adolescente que acabam trazendo os sintomas.

 

\textbf{G}: Sim, os sintomas de bode expiatório dessa realidade
paranoica. Enquanto você estava falando, me lembrei quando eu lancei o
primeiro livro contra a discriminação racial na lei brasileira com
prefácio de Eduardo de Oliveira, que era um pensador negro, na Casa
Afro"-Brasileira com Paulo Matoso.

O primeiro livro se chama ``A Discriminação Racial da Lei Brasileira'',
e uma jornalista perguntou (isso há algumas décadas atrás): ``Mas o
senhor é branco, por que o senhor tá preocupado, já que o senhor é
branco?''. Eu nunca consegui me ver como branco.

 

É exatamente esta fragmentação entre a visão imposta e a auto"-percepção,
a qual se refere Sartre muitas vezes, e que é a condenação para
liberdade. Eu sempre me vejo \emph{outsider}, mas eu quero deixar muito
claro, e é muito importante nessa oportunidade para que não haja dúvida
nenhuma: não com a melancolia da diáspora judaica! Eu não tenho nada a
ver com isso.

 

Isso é para banqueiro. Melancolia para passar o pires e pegar dinheiro
não é comigo, nunca foi. Nem comigo, e nem com a tradição, inclusive
genética minha, nem do meu pai nem da minha mãe. Os dois eram rebeldes
que abandonaram a Polônia atrás de um novo mundo, inclusive contra a
ortodoxia. Meu avô, pai da minha mãe, que era rabino. Optou por ser
\emph{Choichet}. Eis que minha avó se recusava a usar peruca,
obrigatória para esposa de rabino.

 

Eu escrevi um trabalho, ``Não me deixe morrer!'', na Folha de São Paulo,
que foi um escândalo. A~parcela conservadora da comunidade judaica de
São Paulo ficou horrorizada. Aliás, não foi a primeira nem a segunda e
espero que não tenha sido a última vez. Ficou horrorizada! Como é que um
sujeito fala o que eu escrevi a respeito do vovô e do rabino? Mas
escrevi dentro da melhor tradição sim, de rebeldia, que caracteriza o
moiseísmo, o autentico judaísmo.

 

E que caracteriza o extraordinário rabino que foi Jesus. Porque é por
essa vertente do profetismo, ou do Martin Luther King, é por essa
vertente de independência que eu acredito na saída metanoica da neurose
e do persecutório. Eu acho que psicanálise só existe dessa maneira.

 

No Serviço Social do Exército Brasileiro, trabalho que eu defendi lá na
\versal{PUC}, com José Pinheiro Cortez, que depois me levou para o Franco
Montoro, hoje relido é um manifesto revolucionário. O~que na verdade eu
proponho ali, é que o exército deixasse de brincar de guerra contra a
Argentina, e principalmente deixar de brincar de guerra contra o povo
brasileiro como fez na época da ditadura, o senhor Mourão Filho.

 

Através da espada de ouro que foi entregue ao Marechal Lott que deveria
atuar. Ele queria transformar o exército numa grande força de vitalidade
de transformação social no Brasil. Mas, naquele instante, ele foi traído,
como posteriormente Jango foi traído. O~Juremir Machado escreveu
brilhantemente sobre isso no livro dele ``O golpe midiático militar --
1964''.

 

Foi um golpe midiático militar e paranoico. É~significativo que grande
parte da angústia e da aflição e da violência, e do crime hoje no
Brasil, que faz com que quinhentas mil pessoas estejam dentro de
penitenciária. O~que é isso? Colocar quinhentos mil, praticamente todos
homens. Eles deviam ser organizados através de penas para o trabalho,
como propus no ``Direito no Divã''. Criar uma tarefa tipo Roosevelt com
o New Deal nos Estados Unidos. Você resolve os problemas econômicos e
sociais do Brasil em um ano.

 

Não existe vontade política, é muito mais psicológica. Não quero usar
a palavra política; a questão é psicológica. É~o sadismo, é
a crueldade. A~oligarquia brasileira é sádica, e o povo é masoquista.

 

\textbf{R}: Eu quero questionar isso um pouco. Eu venho há quarenta anos
ouvindo que a oligarquia é isso e aquilo. Mas na minha vivência, na
minha percepção e também nas minhas pesquisas, eu percebo que não existe
de fato uma oligarquia brasileira.

 

Se você for ver efetivamente as grandes famílias, de Minas Gerais, da
Bahia ou mesmo do Rio de Janeiro e de São Paulo, os famosos
quatrocentões, esse pessoal na prática existe muito pouco.

 

O que você vai ver são justamente pessoas que chegaram depois. Vemos
famílias de imigrantes que se tornaram ricas, que talvez cresceram
ligadas a antigas famílias mas hoje ostentam nomes alemães, judeus,
árabes, italianos, quem manda hoje. A única exceção seria o Aécio Neves,
que é talvez a última demonstração deste tipo de tradição.

 

O que eu quero dizer é que quando a gente fala da oligarquia
brasileira, dá uma certa noção de que haveria um entrelaçamento de
famílias, que estariam a cem ou duzentos anos dominando a cena
brasileira, e na prática não é isso que acontece. Na prática o que a
gente vê é o Partido dos Trabalhadores chegar ao poder e ter os mesmos
mecanismos de comportamento da tal oligarquia que eles mesmos
combateram.

 

\textbf{G}: Quando eu falo em oligarquia brasileira, talvez a expressão
poderia ser modificada para a oligarquia no Brasil. Talvez a melhor
colocação seja essa, porque realmente a oligarquia não tem pátria.
Grande parte dessas famílias as quais você se refere, na realidade elas
sempre tiveram um pé no Brasil e um pé na Europa. Os filhos sempre foram
educados na Europa. Sempre tiveram uma cabeça de Europa.

 

Seja quem for, nem sei se precisamos citar muitos nomes, mas eu insisto:
a oligarquia no Brasil tem nojo de povo! O que acontece é que existe uma
capatazia: os capatazes da fortuna, que pode ou não ter nascido no
Brasil. É~uma mera questão de coincidência. Porque esse entrelaçamento
do qual você está falando, na realidade existe há quinhentos anos. Começa
com Portugal, com os holandeses, com os franceses, com alemães, com
árabes, com japoneses. Isso é internacional. O~dinheiro é internacional.

 

Teve aquela frase, que eu acho que foi de Engels, que gerou mais uma
polêmica que eu tive, em que ele diz ``o antissemitismo é o socialismo
dos imbecis''. Quiseram dizer que a questão do dinheiro é uma questão
judaica. Isso é uma estupidez. O~Gustavo Barroso dizia que o Armando
Salles de Oliveira, que era candidato a presidente da república, e era
de família de portugueses, era judeu. Era de ascendência judaica, como
grande parte da população brasileira é de ascendência de cristãos novos
de Portugal e você sabe disso.

 

Então, realmente você tem razão, não é oligarquia brasileira no sentido
daqueles nascidos obrigatoriamente no Brasil ou descendente de nascidos.
Isso é completamente indiferente. A~ladroagem não tem uma identidade
comprometida com o conceito de pátria, no sentido da casa de muita
gente. Porque nesse sentido de ``casa de muita gente'' é que eu me
incorporei no grupo nacionalista de esquerda ligada ao Marechal Lott.

 

Qual era a vontade e a intenção? De reagir naquele momento ao brutal
imperialismo norte"-americano. Eu atravessei o Brasil fazendo a campanha
do ``Petróleo é nosso!''. Eu fui candidato a deputado estadual fazendo
dobradinha com o Dagoberto Sales, e só não fui eleito porque na época
votação era feita por cédulas e a roubalheira nas apurações era total e
absoluta.

 

Mas, ainda agora, por exemplo dentro do \versal{PSOL}, claro que eu tenho uma
posição independente. Latu sensu, eu não pertenço a partido político
nenhum, nem a grupo nenhum. Eu militava no \versal{PSOL} por acreditar ser o mais
próximo, mas com bastante distância das coisas mais profundas nas quais
eu acredito. Como antes no \versal{PSB} e mais atrás na \versal{UJC} e no movimento
sionista.

Hoje minha carteira é do ``eu sozinho''. A minha crença é de que as
grandes transformações individuais, deixando bem claro, grupais, sociais
e nacionais, na minha crença, no século 21, vão se estabelecer no campo
do espírito, no campo psicológico! As grandes lutas vão se travar, não
no campo da economia, nem da política, nem no militar, mas na disputa da
mente.

 

Daí a importância, por exemplo, desse trabalho. Nem que seja eu e você
conversando aqui e mais cinco gatos pingados, compartilhando pela
internet, nós estamos num \emph{bunker} revolucionário, discutindo o que nós
estamos discutindo. E~esta é na minha opinião, outra grande conquista
tecnológica do nosso tempo que é a internet.

 

Norbert Wiener em ``God\&Golem, Inc.'' reporta a internet ao pensamento
dialógico e de contestação judeu.%\nota{Esta entrevista está disponível na íntegra em~goo.gl/\versal{F}8lCxS.} 

\fechafala
