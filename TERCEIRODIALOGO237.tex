\chapterspecial{Terceiro diálogo -- Só pobre sofre por inteiro}{}{}
 

No terceiro diálogo, Jacob Pinheiro Goldberg discute o sofrimento e a
angústia psíquica. Seria a doença psíquica uma decorrência de um
desajuste social ou uma impossibilidade da pessoa conviver com ela
mesma?

Segundo Goldberg, o sofrimento psíquico se origina de duas grandes
realidades. Uma, é a fragilidade e a vulnerabilidade do ser humano.
Diante da realidade do mundo, o ser humano é física e emocionalmente
muito pequeno e frágil. Daí sente"-se ameaçado, e do medo decorrem todos
os sofrimentos e as dores emocionais.

A dificuldade das pessoas em enfrentar essa realidade ameaçadora,
perturbadora, resulta num desequilíbrio interno. Por mais que o ser
humano tenha recursos e dotes, não tem preparo suficiente para entender
seu papel diante do real.

Portanto, as pessoas construíram, desde os primórdios da história, as
arquiteturas de proteção. Das mais diretas e simples, como o vestuário,
até os elementos de imaginação, os recursos de invenção, todas as
posturas psíquicas e físicas para conviver com esta aflição.

O processo até agora foi conturbado, complexo, confuso, mas aponta numa
direção de esclarecimento, de apoderar"-se muito lentamente da
inteligência, para entender todos esses fenômenos de exterioridade, e a
repercussão desses fenômenos na interioridade da pessoa.

Para Goldberg, há uma correlação entre a infância, a adolescência, a
idade madura e a senectude, que é: na infância se entende muito pouco,
portanto se fantasia e se inventa. Na adolescência a pessoa descobre que
tem alguma força, algum entendimento e alguma compreensão, e tenta usar
isso como defesa e em seu benefício.

Na idade adulta o ser humano alterna entre se imaginar potente ou
onipotente. O~adulto se imagina em condições de dobrar o próprio mundo.
Neste momento reinventa as armas, a arte, a fé religiosa, e todos os
recursos e manobras para suportar o isolamento diante do Todo, do cosmo.

Na senectude tem a percepção sensorial, intelectual e acionada por todos
os medos, pois definitivamente tem de compreender que essa identidade
vai deixar de existir. E~quais as saídas, então, diante dessa realidade
final? Uma é o desespero, com todas as máscaras que o desespero pode
vestir; e outra, a sabedoria e o mergulho nesse Todo.

Cada pessoa encontra sua forma e suas maneiras, de percorrer essa saga.
De alguma maneira, toda a narrativa e toda a documentação da condição
humana é o registro da experiência pessoal, social e grupal. Essas
experiências humanas abrigam duas realidades realidades conflitantes e
decisivas. Uma, o intuito de construir, outra, o desespero da
destruição.

Para Goldberg, este é o motivo pelo qual não existe psicologia fora do
contexto social. O~escravo tem uma psicologia, o senhor tem outra
psicologia. Toda visão do psiquismo humano, toda a perspectiva está
eivada, viciada, predisposta, pós"-disposta pela sua condição econômica e
social.

Neste sentido, o rico tem condições de sofrer menos. O~pobre sofre por
inteiro. O~despossuído não tem condição de minorar seu sofrimento. O~tratamento psicológico é um paliativo para o sofrimento humano.

Goldberg se pergunta se cada um de nós nasceu aqui como um aborto da
natureza que deu certo, ou é uma representação simbólica da imagem e
semelhança de Deus, que tem um compromisso permanente de se transformar
e transformar o mundo?

O ser individual, somente na posição de não escravo, teria a
oportunidade de escolha real da sua narrativa pessoal, se dentro da
realidade daquele fragmento de vida em que se encontra pudesse
encará"-lo de uma maneira a diminuir o seu próprio sofrimento.

Porém, existem sociedades, como a brasileira, que de alguma maneira
compactuam com a ideia sadomasoquista de exploração do outro,
principalmente a exploração étnica dos negros. Pode"-se opor a isto pela
religião, como foi a marcha no Sinai dos escravos libertos do Egito, com
a emancipação da mulher, do homossexual, do negro, da criança, do velho,
do rico e do pobre, do ser humano, do belo e do horrível. É~só a partir
desse patamar de independência, que o ser humano pode fazer sua opção
moral.

As duas alternativas dessa opção moral são viver a sua vida em
plenitude, inserido no cosmo; ou destruir o outro, se destruir e
destruir o mundo e a natureza.

\begin{center}\asterisc{}\end{center}

\abrefala
 

\textbf{Renato Bulcão}: Na primeira conversa nós falamos a respeito da
fragmentação da pessoa. Na segunda conversa nós falamos a respeito das
possibilidades daquilo que era considerado doença, e especialmente de
uma qualidade especial do ser humano que seria o carisma. Nessa conversa
eu gostaria de fazer uma pergunta que é subsequente: o sofrimento, a
angústia psíquica, ela é decorrência de um desajuste social, como muitos
advogam? Ou ela é uma impossibilidade do ser
conviver consigo próprio?

 

\textbf{Jacob Pinheiro Goldberg}: Certo. O~sofrimento psíquico se
origina de duas grandes realidades. Uma delas, a fragilidade, a
vulnerabilidade do ser humano, tanto física como emocional, diante da
realidade. A~realidade aí, vista como a condição do mundo. Dentro da
condição do mundo, o pequenino ser vivente humano se enxerga, se
percebe extremamente frágil e ameaçado. Daí o medo, e do medo todos os
sofrimentos e dores emocionais, decorrentes.

 

A dificuldade de enfrentar essa realidade ameaçadora, perturbadora, ela
sempre implicou num desequilíbrio. Num desequilíbrio interno. Por mais
que o ser humano tenha recursos e dotes, ele não tem o preparo
suficiente para entender ainda o seu papel diante desse real.

 

De outro lado, como é que as pessoas construíram, desde os primórdios da
história, as arquiteturas de proteção? Das mais diretas e simples, desde
o vestuário, até os elementos de imaginação, os recursos de invenção, as
formas todas mentais e corporais, para conviver com esta aflição?

 

Tem sido um processo extremamente conturbado, complexo, confuso, mas que,
sem dúvida, aponta numa direção de esclarecimento, e de apoderar"-se muito
lentamente de elementos de inteligência, para entender todos esses
fenômenos de exterioridade e a repercussão desses fenômenos na
interioridade da pessoa.

 

Fazendo uma correlação, que é uma das maneiras pessoais minhas, mas que
também é o tecido básico da minha forma de entender e me conduzir na
vida: fazendo, portanto, uma correlação infância-adolescência-idade
madura"-senectude.

 

Fazendo um paralelismo: na infância eu entendia pouco ou não entendia
quase nada, ou não entendia nada. Portanto, eu me limitava a inventar. É~o campo da fantasia e da invenção.

 

Na adolescência eu me debatia. Você descobre que tem alguma força e
algum entendimento e alguma compreensão, e tenta usar isso como defesa e
em seu benefício. Na idade adulta você se imagina entre potência e onipotência.
Você se imagina em condições de dobrar o próprio mundo. Aí você inventa,
não mais lá atrás, o fogo, mas você inventa agora as armas, a religião,
a arte, a fé religiosa, todos os recursos e manobras para suportar o
isolamento diante do Todo, do cosmo.

 

Na senectude tem a percepção sensorial, intelectual e acionada por todos
os medos, de que você definitivamente vai ter que compreender que essa
identidade vai deixar de existir. Desta forma vai deixar de existir. E~quais as saídas então, diante dessa realidade final, perante a realidade
total? Basicamente são duas: uma, o desespero, com todas as máscaras que
o desespero pode vestir; a outra, a sabedoria e o mergulho nesse Todo.

Me parece que mais ou menos este é o mapeamento. Cada pessoa encontra
suas formas e suas maneiras de percorrer essa saga. Essa é uma aventura
sem volta. De alguma maneira, toda a narrativa e toda a documentação da
condição humana é o registro da experiência pessoal, social e grupal.

 

Acontece que dentro de todos esses jogos, existem duas realidades
íntimas do ser humano, que são realidades conflitantes, mas são
decisivas. Uma, o intuito de construir. Outra, o desespero da
destruição. E~você percebe todas as utopias políticas, sociais, que
aconteceram no século vinte de uma maneira dramática, como uma grande
ópera. Talvez por causa do cinema e principalmente, depois pela
televisão, em termos de imagética.

 

Existe um instante que sempre me impressionou muito. Quem fala desse
instante muito bem é uma descrição feita pelo escritor Isaac Deutscher
em ``Profeta Armado'' ou na biografia do Stalin. De qualquer maneira, é
nesses estudos em que ele conta uma cena que me parece muito
significativa. O~Politburo da União Soviética estava reunido. Estavam
presentes Trotsky e Stalin. Um dos dois iria ser escolhido naquela
reunião secretário geral do Partido Comunista da União Soviética. O~partido, que acenava com a possibilidade de uma sociedade com o mínimo
de dor. Era a promessa da utopia messiânica que o comunismo, naquele
instante, prometia. Os dois estavam presentes: de um lado o seminarista,
Stalin, de outro lado o intelectual, Trotsky. Os dois discutindo, e a
partir de certo instante fica evidente que a escolha iria recair em
favor de Stalin, não obstante o testamento político de Lenin que
apontava na direção de Trotsky.

 

Trotsky, impulsivo, genioso e genial, hipersensível, mas um homem
profundamente ligado à generosidade, solidariedade, compreensão,
portanto construção, se aborrece profundamente quanto tem a percepção de
que quem vai ser escolhido é Stalin, o indivíduo violento, o sujeito
arbitrário e, portanto, perigoso. Ele se levanta abandonando a reunião,
com a impressão de que ao tomar aquela atitude, os camaradas se
levantariam e iriam atrás dele. Acreditando que aquele gesto teatral
iria marcar historicamente o retorno.

 

Bom, Trotsky se levanta e de maneira teatral se dirige para a porta,
abre a porta e sai tentando bater a porta. A~cena é muito bem descrita
por Isaac Deutscher. Acontece que as portas dos palácios russos não
estavam azeitadas. E~a porta se fecha lentamente, criando um anticlímax.
Aquilo que poderia ser o impacto, o abalo sonoro marcante, com todos os
significados de ruptura, acaba se transformando em um movimento quase
chaplinesco e quase grotesco. Um movimento menor na grande cena
histórica.

 

Acontece que a história é feita sempre pelos movimentos menores e pelas
possibilidades não realizadas. E é nesses movimentos menores que reside
grande parte do sofrimento humano. Desde a época da ditadura no Brasil,
que se discute a questão da distribuição da riqueza. O~Brasil tem a
sétima, oitava economia do mundo. Só que a distribuição da riqueza no
Brasil, ela equivale mais ou menos ao centésimo lugar no ranking
internacional. Sob o ponto de vista político, a questão da miséria no
Brasil poderia ser resolvida em trinta dias. É~questão de dias.

 

Nós já tivemos aqui um estudioso, que de forma irônica disse que a
constituição da República deveria ser só um artigo: ``Todo brasileiro é
obrigado a ter vergonha na cara. Parágrafo único: ficam anuladas as
disposições em contrário!''. Realmente isso é verdade. As classes
dirigentes no Brasil não têm vergonha na cara e não têm inteligência
suficiente pra resolver o grande problema, do maior sofrimento do povo
brasileiro. De onde nós viemos e onde nós estamos. Nada justifica
absolutamente nada em termos de economia: o desemprego, a fome e as
condições de miserabilidade.

 

O país é muito rico, ele não é simplesemente rico, ele é muito rico! Alguém então pode
perguntar: se é tão simples assim, por que não se faz? A resposta me
parece também muito simples. Por perfídia, meramente por perfídia, por
sadismo. O~ser humano tem no seu psiquismo o traço sádico, que é traço
destrutivo, que é vontade de mutilar, torturar, matar. De ter gozo no
sofrimento alheio. Então, tentando responder a sua pergunta, é a
conjunção desses dois fatores, que, na minha opinião, agravam o sofrimento
humano. Portanto, existe o sofrimento, que é impossível de ser resolvido,
que está ligado à própria natureza da nossa condição. Tem também o outro
sofrimento, que é o sofrimento construído sistematicamente pela intenção
de destruir.

 

\textbf{R}: Tentando ver a sua resposta por um outro ângulo, eu tenho os
seguintes elementos: o sofrimento individual do homem comum nasce a
partir da dificuldade de acompanhamento da sua própria narrativa de
vida.

 

Voltando um pouco atrás, essa narrativa de vida não é coesa, ela é
múltipla, na medida em que a pessoa é constituída de fragmentos de
histórias, que ele vive ao mesmo tempo, com as diferentes pessoas e
ambientes com quem ele trata. Nesse sentido haveria um contexto, e esse
contexto acaba direcionando a própria vivência da história individual,
que é o contexto histórico.

 

Contexto maior no qual a pessoa tem a sorte ou azar, de nascer e viver,
construindo portanto, as suas narrativas. Dentro desse contexto
histórico, além de eventuais traços sádicos de pessoas que estão no
poder, mais importante seria efetivamente a ideia da possibilidade de
como criança, adolescente, homem maduro ou idoso, seria poder a pessoa
atentar às narrativas dos problemas menores. Porque justamente dos
problemas menores é que sairiam as escolhas para a continuação das
narrativas.

 

Justamente esses problemas menores, que me pergunto se são mais
cotidianos ou se são mais extemporâneos, que eventualmente determinam as
escolhas para a continuidade da narrativa, que tornariam a vida da
pessoa um pouco mais angustiada ou aflita? Ou não? Qual é a ideia?

 

\textbf{G}: Quando eu falei da questão do sadismo, eu quero deixar bem
claro, que a minha concepção de sadismo engloba o traço masoquista. Até
um determinado instante se imaginava que o sadismo e o masoquismo
seriam quase as manifestações opostas do psiquismo. Não são. Elas
compõem o mesmo núcleo de sofrimento, de dor, tomando só características
e especificidades diferentes.

 

Explicando melhor: Abel, ele quer morrer. Abel municia Caim com a arma
que vai matá"-lo. Portanto, ele é cúmplice do fratricídio. É~exatamente
isto. O~papel que a mulher acaba exercendo de submissão, diante da
ditadura masculina; o papel que os grupos humanos em geral acabam
estabelecendo vítimas e algozes, é um cenário que tem que ser
desmontado. Precisa ser desmontado o tempo inteiro.

 

Só existe o carrasco, porque o prisioneiro fica dentro da cela. Hoje,
enquanto nós estamos aqui conversando, eu calcularia que pelo menos dez
por cento da população brasileira vive em condições absolutamente
sub"-humanas. Em termos de saúde, em termos de escolarização, enfim, em
todos os termos. O~sujeito desfila com um automóvel Mitsubishi pela
Avenida Paulista. Tem uma criança famélica na esquina pedindo esmola.

 

A construção mental do dono do Mitsubishi é: essa criança está sendo
manipulada pelo pai que é um vagabundo que não quer trabalhar. Eu sou um
sujeito privilegiado porque eu trabalho, então Deus me escolheu pra ser
milionário. Esta arquitetura faz com que eventualmente essa criança na
adolescência venha matar esse dono do Mitsubishi, que por sua vez, fará
de tudo para prender o futuro bandido, baixando a idade da maioridade
penal. Primeiro para 16 anos, depois para 14 e nos Estados Unidos em
alguns estados praticamente não existe esse conceito. Uma criança que
pega a arma do pai e dá um tiro no irmão com quatro anos de idade é
condenado a prisão perpétua.

 

Aqui não precisa ser condenado a prisão perpétua, porque a própria
estrutura da sociedade vai matar essa criança, ou de fome ou de acidente
de trânsito. Para dar um jeito de liquidar através de um genocídio, que
tem muito de etnocídio contra o negro. Porque é em cima desse etnocídio,
que nós temos a prevalência do branco no Brasil.

 

E tudo isso é feito com nossa cumplicidade. Eu ouço muito pessoas
dizendo que na Alemanha se matou seis milhões de judeus, o que seria
inexplicável. Fica essa discussão essa polêmica vazia: os alemães sabiam
ou não sabiam o que estava acontecendo?

 

Aí eu pergunto: e os brasileiros, eles sabem ou não sabem que tem
quinhentas mil pessoas presas em penitenciárias quase em condições de
campos de concentração? Sabem sim! Todo mundo sabe! Político sabe, a dona
de casa sabe, quem está andando na rua sabe. Quem é que não sabe que a
empregada doméstica que vai na sua casa lavar roupa, ela vive
muito pior que um escravo da senzala? Vive muito pior! Ela passa quatro
horas na condução para vir, e quatro horas para voltar para a casa dela.
É~condição de escrava mesmo! Não obstante isso, as nossas colunas
sociais da nossa mídia, que se diz libertária, que se diz democrática e
republicana diariamente (e nós somos obrigados a engolir este gênero de
manipulação de mentira) afirma que nós vivemos em um Estado de Direito.

 

Direito do quê? Eu escrevi um livro, ``O Direito no Divã'', organizado
pelo meu filho Flávio. Ele sugeriu que a gente dedicasse o livro ao Luiz
Gama, o advogado filho de escravo, que defendeu o direito
do escravo, na hipótese do dono dele se recusar a libertá"-lo, de
matá"-lo, alegando que o escravo não tem vida.

 

Hoje no Brasil nós temos alguns milhões que realmente não têm vida. O~sujeito que vai procurar socorro no \versal{SUS} e nos chamados melhores
hospitais de São Paulo, encontra por trás até destes melhores hospitais,
a indústria farmacêutica. Esta indústria vai fazer você usar o remédio
que mais vai dar lucro. Servindo ou não a sua saúde!

 

\textbf{R}: Eu quero voltar ao início. Isso significa que
obrigatoriamente todas as decisões de qualquer pessoa, seja aquela que
acredita que Deus a escolheu pra ser rica, seja aquela que acredita ou
não em Deus, mas de alguma maneira vive uma vida sofrida e sem a menor
oportunidade, será que todas as escolhas e todos os problemas menores
que acabam ditando o eixo das suas narrativas de vida?

 

Eles são obrigatoriamente inseridos num contexto social? Colocada a
questão de outra maneira; será que existe psicologia fora do contexto
social?

 

\textbf{G}: Não existe psicologia fora do contexto social. O~escravo tem
uma psicologia, o senhor tem outra psicologia. Começa que o senhor faz
psicanálise; o escravo engraxa as botas. Toda a psicologia, toda a
abordagem, e toda a visão do psiquismo humano, toda a perspectiva está
eivada, viciada, predisposta, pós"-disposta pela sua condição econômica e
social. Só que isso não é o todo. Isto é pré-requisito.

 

Igualdade social não é nem sequer um programa marxista, é da
Constituição da nossa república! Todos os direitos estão inseridos na
nossa Constituição. O~sujeito tem que ter direitos básicos pra ele ser
uma pessoa. Senão, ele não é uma pessoa: ele é um escravo! O resto é jogo
de cena de uma cultura que é totalitária. Se não quiser usar esta
palavra com alguma conotação política, use totalista, que no final é a
mesma coisa. É~o uso do ser humano em benefício de outro ser humano, ou
de grupos. Esse é o pré-requisito. Só depois de resolvido esse pré-requisito,
e nesse aspecto a minha concepção é absolutamente marxista,
só depois de resolvido isso é que nós passaremos a segunda etapa. É~a
etapa do entendimento para suavizamento, das angústias e das aflições do
sofrimento humano.

 

Só rico tem condições de sofrer menos. O~pobre, não, esse sofre por
inteiro. O~despossuído não tem condição. Às vezes se usam aí expressões
do gênero: conquista da cidadania. Eu tenho acompanhado por décadas os
dois processos. Desde muito cedo, eu tive consciência de uma intenção
que eu acredito que eu levei à prática. Eu tive a intenção de inscrever
na sociedade brasileira e no imaginário da sociedade brasileira a
psicologia como uma das ciências de saída pra crise do humano.

 

Eu acho que eu fiz isto, através da grande mídia, \versal{TV} Globo, Estadão,
Folha de São Paulo, os principais jornais e os principais instrumentos
de comunicação. Durante o tempo que eu pude e com circunstâncias que
foram permitidas. Eu tentei sempre ocupar o nicho, o \emph{bunker}, com a
absoluta consciência de que os lugares eram poucos, e por pouco tempo.
Mas eu hoje tenho absoluta informação, através até da reação da opinião
pública, de que eu exercitei esse papel.

 

Eu comecei na \versal{TV} Tupi atingindo sempre as grandes massas. Eu acho que eu
consegui usar uma linguagem direta e transparente para as grandes
massas. Consegui usar uma linguagem científica e trabalhada, para
aqueles que tinham recursos de assimilação e de gestão das ideias. Esse
é o papel que escolhi, que tenho escolhido e que tenho exercitado. Eu
sempre tive consciência que se tratava de uma opção revolucionária. Essa
sempre foi a minha posição. Dessa maneira, eu passei todos esses anos.
Nunca ao largo, sempre fazendo um esforço permanente de participação. Em
cada episódio da história brasileira.

Isso desde a poesia até o esporte. Eu me lembro de uma das vezes em que
eu fui a um programa de televisão e o apresentador, por sinal um sujeito
que era um capacho de direita na televisão, tentando criar um
constrangimento para mim perguntou, desqualificando minha participação:
``É, doutor Jacob, o senhor acaba de fazer essa exposição, mas o senhor
seria capaz de participar do quadro com as mulatas do Sargentelli?''. Eu
disse: ``Não só sou capaz de participar, como vou dançar com elas''. E eu
dancei com as mulatas do Sargentelli diante da televisão brasileira.
Para escândalo do conceito de escola enclausurado dentro das
universidades, tentando mudar o mundo e transformar o mundo com
discursos de esquerda e políticas de direita!

Eu me neguei a esse papel e me nego esse papel. Mentalmente,
filosoficamente e corporalmente eu tenho a minha opção. A~minha opção é
a psicanálise, tem um compromisso na minha opinião, como a religião
deveria ter, como filosofia deveria ter, como o saber deveria ter, com a
dor e sofrimento humano.

Tratamento é só isso. É~paliativo para o sofrimento humano. De que
maneira isso acontece, se através de aplicação de agulhas pela
acupuntura, se através da catarse, se é por meio de lacanianos,
freudianos, junguianos, ou seja, quais forem, tudo isso é menor. Se a
assunção disso se fará através de uma bandeira, de uma ordem, ou de
outra ordem, na minha opinião isso não é o ponto nodal da questão.

 

O ponto nodal da questão, sempre, o tempo inteiro é: cada um de nós
nasceu aqui como um aborto da natureza que deu certo? Ou cada um de nós
é uma representação simbólica da imagem e semelhança de Deus, que tem um
compromisso permanente de se transformar e transformar o mundo?

 

\textbf{R}: Só uma última questão pra gente fechar esse tema. Isso
significa então que o ser individual, somente na posição de não
escravo, tem a oportunidade de escolha real da sua narrativa pessoal?

 

Portanto, ele teria, como o senhor mesmo usou a palavra, a possibilidade
de dentro da realidade daquele fragmento, encará"-lo de uma maneira a
diminuir o seu próprio sofrimento. Porém, existem sociedades como a
nossa que de alguma maneira compactuam com a ideia sadomasoquista de
exploração do outro.

 

No caso mais específico do Brasil, de uma exploração étnica, dos negros.
Nesse quadro geral, permitem"-se fazer escolhas de narrativas, que podem
ser através da igreja, ou através dessa ou daquela forma de terapia.
Excluem assim qualquer outra possibilidade, até mesmo de coexistência,
com essa outra parte da sociedade. De alguma maneira socialmente
grandiosa eles os excluem sistematicamente?

 

\textbf{G}: É isso! Eu acho que você fez a síntese exatamente do
conceito da libertação. Seja através de religião, a teologia da
libertação, seja através da marcha no Sinai dos escravos libertos do
Egito, seja da libertação e da emancipação da mulher, do homossexual, do
negro, da criança, do velho, do rico e do pobre, do ser humano, do belo
e do horrível. É~só a partir desse patamar de independência, que o ser
humano poderá fazer a sua opção moral. Quais são as duas alternativas
diante dessa opção moral: viver a sua vida em plenitude, inserido no
cosmo; ou destruir o Outro, se destruir e destruir o mundo e a natureza.
O~suicídio do insensato.

\fechafala 
