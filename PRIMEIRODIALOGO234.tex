\newcommand\abrefala{\begingroup\parindent0pt \setlength{\parskip}{6pt plus 3pt}}
\newcommand\fechafala{\endgroup}

\chapterspecial{Primeiro diálogo}{}{}
 \epigraph{``A mentira é uma graça''}{Fanny Goldberg.} 

 

Neste primeiro diálogo, Jacob Pinheiro Goldberg decide falar de sua
teoria psicanalítica, sem abrir mão de se referir às suas vivências, que
servem de base para sua forma de pensar. Seu trabalho é denominado por
ele de ``Psicologia imagética''. Ele configura sua forma de trabalhar como
dramaturgia psicológica. Essa dramaturgia psicológica cria um contexto
tanto na terapia de grupo, quanto na terapia individual, mesmo na
atividade social e comunitária.

Para Goldberg, nós somos seres que representam. Em cada situação agimos
de forma diferente, mas que significa alguma coisa importante para nós.

Neurose e psicose seriam expressões do desconforto da tentativa de uma
pessoa seguir um comportamento monolítico, adotado a partir de cobranças
externas. Quanto mais a pessoa representar aquilo que ela não é, mais
ela se aproxima de uma imposição de ser aquilo que ela não é. Isso gera
angústia, que cresce se tornando uma neurose, ou no limite de um
comportamento doente, uma psicose.

O ser humano normal, para Goldberg, atua de forma fragmentada a partir
de sua relação com aquilo que ele percebe na realidade. Para ele, não há
uma essência da pessoa, mas uma resposta da pessoa àquilo que ela está
vivenciando. Por isso, a atuação do psicanalista neste sentido, deve
examinar a imagem que as pessoas estão fabricando o tempo todo, para
determinarem a sua aparência. É~através dessa imagem que as pessoas
conseguem se representar no mundo. Para tanto, elas utilizam diversos
recursos, como máscaras e artifícios, que devem servir para se
empoderarem e intervirem na ordem das coisas.

Assim, a hipocrisia é uma representação burlesca e é uma farsa
empobrecida. Os medíocres são hipócritas e os medíocres são perversos,
dementes e sádicos. A~doença é hipócrita. Quando o sujeito se aproxima
do seu papel na cena, ele vai se afastando da hipocrisia e vai se
afastando do Nada. Ele passa a acontecer. Na medida em que o indivíduo
participa da cena que ele percebe como realidade, com uma maior
percepção e maior empenho, ele recupera a sua essência.

O ser humano busca um sentido para a sua vida. Nós queremos justificar
nosso papel no grande espetáculo. Mas primeiro precisamos saber qual é o
nosso papel. Então as pessoas buscam um papel através das intermediações
oferecidas pela religião, pelas ideologias, pela mídia que as levam às
mistificações.

Por outro lado, a loucura é o estado de espírito mais trágico da
condição\textbf{}~humana, porque é a perda da ego"-centralidade. O~sujeito deixa de ser ele mesmo, para ser tomado. Mas o inverso, a
ordem integral, é outro engano. A~estética da ordem nos dá a sensação
ilusória de poder sobre a realidade. Mas isso é uma fantasia. Nós
precisamos aprender a conviver com as diversas realidades que
percebemos. Precisamos aprender a conviver com os nossos fragmentos, sem
pretendermos criar um ser único e íntegro, pois isso é uma fantasia.

A psicanálise, assim como as demais terapias, fazem parte da cultura. O~que se faz num consultório é o encontro de duas pessoas, em busca de um
momento que permita recuperar a alegria de viver. Viver é aceitar a
travessia entre o nascimento e a morte. Mas as pessoas sentem medo, e se
perguntam muitas vezes se vale a pena viver, e isso causa desespero.

A maioria das pessoas ainda está na caverna de Platão vendo sombras e
imagens. Por isso que Goldberg defende a psicologia imagética, para
interferir neste processo, interpretando e transformando.

\begin{center}\asterisc{}\end{center}

\abrefala

\textbf{Jacob Pinheiro Goldberg:}~Eu começaria já com um conceito de
filosofia que eu sei que você\textbf{}~aprecia. Eu acho que a gente vai
calcar mais nesse trabalho aqui na teoria, porque elemento biográfico a
meu respeito já tem gente escrevendo. Podemos permear a minha vivência
com a teoria, e devemos. Mas a teoria não deve ser a linha condutora,
como você sempre insistiu e eu resisti. Eu acho que tem que ser mesmo o
meu pensamento. Como é que esse pensamento se forjou e a importância que
ele tem, ou não tem.

 

\textbf{Renato Bulcão:}~Tem alguma diferença neste sentido, ter um
pensamento no palco, ou na mediação de psicodrama?

 

\textbf{G:}~Quero deixar claro que talvez a melhor expressão seria
dramaturgia psicológica e não psicodrama. Porque psicodrama está muito
ligado ao trabalho do Jacob Levy Moreno. Para aqueles que já foram
pacientes ou clientes, sabem que eu não me reporto aos cânones do
psicodrama, mesmo porque discordo do trabalho de Moreno, na
integralidade e sua ``autobiografia'' é suficiente para explicar esta
divergência umbilical.

 

\textbf{R:}~E se eu mudar para terapia de grupo? Será que a sua terapia
de grupo e a\textbf{}~sua atuação no palco têm similaridades?

 

\textbf{G:} Ambas têm essa dramaturgia psicológica. Na verdade, a minha
ideia sempre, é a ideia da representação, a ideia da imagem. Não apenas
trabalhando como psicólogo, mas também ao pensar e ao viver.

 

Eu acredito que nós somos seres que representam. Fundamentalmente, se
por acaso existe uma peculiaridade no meu pensamento, ou pelo menos um
elemento que para mim define muito o papel da existência, é a questão da
representação. Ou seja, nós não somos ninguém em essência. A~ideia da
busca da essencialidade é uma ideia totalista.

 

E é isso que de certa maneira conduz até Heidegger, e seu nazismo tão
simulado, por exemplo. A~preocupação totalista é o fundamento da
psicose. E~de uma maneira mais amena, da neurose. A~superação da neurose
é a libertação desse conceito de busca da integridade. Nós não somos
inteiros, nós somos pedaços. E~a gente monta esses pedaços para
representar papéis. Isto é teatro. E~a vida é permanentemente uma cena.
Mesmo quando você se pretende íntegro.

 

Ou seja, totalista ou totalitário, esse mais do que ninguém, representa
um papel. É~um triste papel, porque é um papel mórbido, e por isso
destrutivo para o próprio indivíduo, e para o mundo.

 

\textbf{R:}~Quer dizer, essa ideia que nós identificamos como kantiana,
da integridade,\textbf{}~da retidão de caráter, da pessoa virtuosa, e
não um virtuoso aristotélico, mas o virtuoso kantiano; que é ideia de
que ``cada ação que você vai fazer, faça como se fosse um exemplo que
você gostaria que fosse seguido por todos os outros''.

 

Essa ideia então ela termina no século 20 ? Agora no século 21, se a
gente quiser pensar a realidade das pessoas, não pode mais pensar dentro
destes parâmetros, que você chama de totalista, e que você identifica
com uma espécie de prisão. Prisão no sentido da psicose e da neurose.
Essa tentativa de ser único e ser total, a potência em última estância,
são a origem do sofrimento?

 

\textbf{G:}~A ideia da psicose e da neurose e eu diríamos que também, da
ritualística,\textbf{}~da inteligência como prisão. Porque antes e acima
de tudo, esta é a prisão. Este é o bezerro de ouro, este é o ritual,
este é Aarão. E~quando Moisés desce, e na minha opinião o momento mais
dramático daquela revelação é a questão da ``máscara'', de ``Moisés com
máscara'', esse é o grande momento!

 

Inclusive se aventa a hipótese de que já não era Moisés. Que Moisés
teria sido assassinado e desce outro personagem. Isso foi levantado
inclusive por cabalistas na Idade Média. Que já não era mais Moisés. Ou
melhor, era muito mais Moisés, em não sendo aquele. Porque Moisés teria
sido sacrificado. Essa questão do sacrifício, ela é muito importante.
Porque o totalismo sempre precisa de um bode expiatório. Precisa sempre
do sacrifício.

 

E aí a gente vê uma linha impressionante que passa, nessa hipótese por
Moisés, mas passa sempre também pela extradição da paternidade com
Rômulo e Remo. É~a loba que é a mãe, a loba que seria a prostituta
romana. Então a própria ideia da civilização romana fica por cima dessa
pedra.

 

E não por acaso, mais tarde, em todos os conflitos e disputas, o império
romano acaba, talvez, num dos mais tragicômicos atores da modernidade
que foi Mussolini. Ele talvez tenha sido o mais engraçado e desgraçado
ator da modernidade. Só superado pela figura grotesca de Hitler e de uma
maneira muito enviesada, por outros ditadores.

 

Quanto a Hitler, Mussolini e os nazistas, eu os vejo como animais de
caça perversos, sem personalidade, portanto fora do humano. São os
capetas genocidas bestiais que devem ser banidos.

 

\textbf{R:}~Deixa"-me voltar um pouco aqui à Moisés. Não toda ética, mas
com certeza toda moral\textbf{}~que forma a chamada civilização
judaico"-cristã está baseada nas leis de Moisés. A~lei cresce a partir
dos Dez Mandamentos, principalmente não dos mandamentos ligados a Deus,
mas dos mandamentos ligados à humanidade, ao ser humano.

 

\textbf{G:}~Aos mandamentos que seriam de Noé, a ideia é essa. Porque a
ideia seria\textbf{}~de que os mandamentos teriam sido de obrigações
para os judeus, ou seja, o jugo seria maior. Os mandamentos de toda a
humanidade, que seriam os mandamentos de Noé, que Deus poupou através da
arca. Mas você estava perguntando…?!

 

\textbf{R:}~Como toda moral nasce nos Dez Mandamentos, e como a
observância aos\textbf{}~mandamentos pode em última estância trazer a
ideia justamente do fascismo, quer dizer, de Mussolini, e relembrando
que no início você disse que nós somos seres fragmentados, sem essência
e que atuamos a partir da relação estabelecida imediatamente, é obvio
que essa proposta de visão se estende também por todo o passado.

 

Apesar de ser uma visão contemporânea, imediata, é muito tentador
pensarmos o próprio Hitler, o próprio Mussolini, desta maneira. As
pessoas não eram só nazistas, elas eram também amantes, estavam
apaixonadas, cuidavam com carinho do seus filhos, enfim, as pessoas
sempre foram fragmentadas.

 

No fundo é isso que eu estou tentando resgatar do seu pensamento e da
sua ideia: na medida em que as pessoas sempre foram fragmentadas, como
esses fragmentos podem se unir para tornar a pessoa saudável, sã ou em
paz consigo mesmo? E como esses fragmentos de alguma maneira acabam
construindo um tipo qualquer de doença?

 

\textbf{G:} Em primeiro lugar esses fragmentos não são peças que podem
ser armadas como se fossem um jogo. Nós fazemos parte de uma
fragmentação universal. A~psicanálise freudiana e as suas decorrências
escorregam nisso também.

 

Numa preocupação, vamos chamar de saúde (que sempre corre o risco da
eugenia). Não existe esse conceito de integridade; nós somos pedaços e
nos conformarmos com a consciência dos pedaços e da fragmentação,
simplesmente para nós podermos viver.

 

Essa ideia de harmonia é o que conduz ao totalismo. Moisés não era um
homem (no sentido de ser apenas um homem). Jesus também não. Não existe
nenhum momento de maior fissura interna, que você possa conceber do que
os momentos epifânicos, por exemplo, de Jesus na cruz. ``Pai, pai por
que me abandonaste?'' Ele está falando de quê? Ele esta falando de José.
Na infância, está nos Evangelhos, a família já corria atrás de Jesus que
estava rezando na sinagoga e dizia: ``Meu filho, por que você abandonou
a sua casa?'' E Jesus respondeu: ``Eu estou na minha casa''. E~ele
estava ali, na consciência do seu desamparo, do seu abandono; a assunção
absoluta do seu abandono. Que é o que ele repete na cruz.

 

Moisés, quando pede a Deus que o poupe da morte, ele não é um herói
druida. Ele não é um herói celta e muito menos um herói ariano, com uma
espada valente, guerreiro valente, desafiando o mundo. Ele é um velho
coitado, deitado no chão, e pedindo, implorando a Deus que o poupe da
morte.

 

São essas figuras extremamente frágeis que têm o vigor que você
levantou, de embasar toda nossa cultura e toda a nossa civilização.

 

Inclusive eu gostaria de dizer para você, que tenho pensado muito numa
questão que eu acho fundamental, que é a questão da Palavra. Você sabe
muito bem que a palavra é um instrumento político, antes e acima de
tudo. Eu acredito que está mais do que na hora de nós começarmos a
questionar o vocábulo ``cristianismo''.

 

Cristianismo é uma religião forjada por Paulo, que tem seu lugar. Mas
Jesus faz a intermediação entre essa religião filha e a religião mãe.
Portanto eu creio que a gente poderia considerar o cristianismo uma
forma de neo"-judaismo.

 

\textbf{R:}~Mas é considerada.

 

\textbf{G:}~Mas com muita resistência ainda. Se tenta criar uma ruptura
dramática entre a fonte e o seu prosseguimento. E~outra coisa que eu
creio que é importante, já que estamos discutindo, sempre com alguma
experiência na área da psicologia e da imagética, que é a área a que eu
também tenho me dedicado: A questão da psicanálise, criada como você
sabe, praticamente no período áureo que coincide com a ascensão nazista
e na tradição do império austro"-húngaro, antissemita.

 

Freud teve muita preocupação de fazer o vínculo com a cultura grega ---
o Édipo --- e assim por diante. Mas ele não conseguiu, na verdade. Eu já
escrevi um trabalho ``Freud e o Judaísmo'', onde digo que as raízes
básicas da psicanálise são cabalísticas.

 

É mais do que tempo de nós enfrentarmos os preconceitos estúpidos e
desarrazoados, que tentam desvincular a ideia da psicanálise do
judaísmo. Tal qual o cristianismo, a psicanálise é uma forma do pensar
judeu, ancorada nos profetas e Moisés. E~mais do que tudo, e aí sim
rompendo tabus, ancorada no Mistério. O~Mistério é o diálogo com Deus.
Um pastor chegou a dizer que Freud era o mais religiosos dos ateus.
Realmente, isso procede. A~própria ideia do \emph{sentimento oceânico}
que Freud fala, é a ideia da percepção da eternidade.

 

\textbf{R:}~Especificamente em relação a essa questão, porque tem mais
duas\textbf{}~pendentes: Na primeira questão, da teatralização. Moisés
teatraliza, Jesus teatraliza…

 

\textbf{G:}~Não por acaso, uma das peças mais interessantes, modernas,
do teatro\textbf{}~contemporâneo, é Jesus Cristo Superstar.

 

\textbf{R:}~Eles teatralizam, e nós sabemos que todos os
atos\textbf{}~públicos são fortemente teatralizados e alguns até
minuciosamente planejados, como fizeram Hitler e Mussolini, para
surtirem algum efeito emocional.

 

A segunda questão, que perpassa a questão da teatralização, é a questão
da harmonia. A~questão da harmonia, a menos que eu esteja enganado, é
fortemente defendida por Pitágoras no momento em que ele diz que tudo
são números. Ele descobre a relação das notas musicais e a partir dessa
relação, pensa em números em harmonia.

 

Obviamente toda essa harmonia, todo teatro, para ser eficaz no
envolvimento da plateia, tem de criar sua própria harmonia. Quase como
se a harmonia fosse sua própria ética.

 

Por último, ``Moisés e Monoteísmo'', que em princípio foi considerado,
digamos assim, o tributo que Freud oferece a religião. Nesse tributo ---
não sou um grande especialista em Freud --- sempre se faz uma alusão a
``Totem e Tabu''. Quer dizer, ``Totem e Tabu'' seria o que ele pensa, e
``Moisés e o Monoteísmo'' seria a desculpa que ele pede para dizer o que
pensa. Eu gostaria que você comentasse um pouco essas ideias.

 

\textbf{G:} Você levanta alguns tópicos que são cardeais. Mas Hitler e
Mussolini não teatralizaram, vampirizaram. A~questão do Número no
Judaísmo, a ideia essencial de Deus está ligada e você colocou bem a
questão de ``Moisés e o Monoteísmo'' e do Freud -- a Unidade.

 

Há um Deus só, e por sinal uma das lendas mais interessantes que existem
a respeito da oração fundamental não só do Judaísmo, é a prece que de
alguma maneira está por detrás de todas as orações chamadas cristãs:
``Ouve Israel, o Senhor é Um, o Senhor é nosso Deus.''.

 

Muito bem, quando é que surgiu essa prece? Segundo a lenda, e talvez
essa lenda seja algo de anúncio do mais extraordinário em termos de
radiação: Jacó estava morrendo, seus filhos estavam preocupados
percebendo a aflição de Jacó. Seu filho mais velho se reúne com os
irmãos e diz: ``O velho está morrendo, e a preocupação dele, é que
depois que ele morra, a gente vá adorar outros deuses. Só que ele está
surdo, e nós temos que informá"-lo da nossa lealdade a um Deus.'' Eles se
reúnem e gritam essa oração na mais potente elaboração que podem fazer,
em termos de voz para que Jacó ouça então eles gritam: ``Ouve Israel
(que é o nome de Jacó), o Senhor é nosso Deus!''.

 

Nós sabemos que existe uma lei de física que informa que nenhuma palavra
desaparece. Elas ficam registradas e reverberando para sempre. Tudo que
nós estamos dizendo agora aqui estará registrado. Estará registrado no
Todo.

 

Então qual é o conceito? Os judeus ortodoxos dizem que é preciso rezar
isso o dia inteiro, porque cada vez que você reza você dá mais vigor às
frases que foram ditas. Jesus por sua vez, ele diz, ``Eu vim para não
trocar uma vírgula da Torah''. A~tradução, depois nos Evangelhos é da
``Lei'', mas aí fica muito anódina. A~lei que existia era a Torah. Então
informa que ele também está informando isso. Nós sabemos, desde o Sermão
da Montanha, que as declarações mais passionais de Jesus são sempre as
declarações de amor a Deus.

 

Quando antes de morrer, Freud escreve e reescreve ``Moisés e o
Monoteísmo'' seis vezes, porque o original foi reescrito seis vezes,
para dar ideia da importância que ele atribuía a esse escrito, é quase
como se fosse um testamento. É~daí aquele vínculo que eu acredito que
existe entre a psicanálise e o judaísmo. E~alguns dizem ``Mas ele então,
um judeu heterodoxo…''.

 

Isso é bobagem, ele é ortodoxo… heterodoxo. São esses tontos
fantasiados, no Brasil, país tropical, com um calor imenso desses, com
roupas da Idade Média, imaginando na ritualística de Aarão; esses sim,
adoradores do bezerro de ouro, acreditando que eles são herdeiros do
judaísmo. O~judaísmo não é definitivamente uma religião. Ele é uma
revolta.

 

Moisés não libertou um povo. Ele libertou um bando de escravos. Tanto é,
que junto com os judeus foram também muitos egípcios, que resolveram
optar pela liberdade. O~melhor judaísmo nunca teve nada a ver com
ortodoxia, principalmente agora na Idade Moderna (o melhor judaísmo
hoje, está em Nova Yorque, não está em Israel). Mas o sionismo é o
marcador ético da história contemporânea.

 

\textbf{R:}~Eu perguntei uma vez a uma pessoa muito inteligente chamado
Moisés\textbf{}~Baumstein, o que era Judaísmo? Ele disse que é uma
maneira de se comportar: uma maneira de ser. Ele tinha certeza que na
China haviam judeus que nós ainda não conhecemos.

 

\textbf{G:}~A minha posição é até um pouco diferente. Eu diria a você
que a maioria\textbf{}~dos chineses são chineses sim, e são judeus sem
saber que são. Aliás, só de curiosidade, na maçonaria nós sabemos que
isso também acontece. A~maioria das pessoas são maçons sem ter
consciência disso. Tanto é, que quando elas passam a se filiar à
entidade maçônica, já estão dando um passo de exterioridade. Mas um
maçom praticamente nasce. O~judeu nasce. Não existe ninguém no mundo
mais judeu do que Gilberto Gil. Eu escrevi isso no livro ``Judaísmos
Ético e não Étnico'', e ele mesmo falando do cantor negro Bob Marley:
``Bob Marley morreu, porque além de negro, era judeu.''.

 

Então, essa ideia do judaísmo como propriedade de um grupo racial, essa
ideia é mais uma ideia totalista. Do mesmo jeito que Jesus rompe isso
quando ele diz ``a minha função, o meu papel é levar o Conhecimento''. O~que basicamente ele tenta fazer é informar, como Moisés tenta informar.
Moisés vai até o Monte, desce do Monte e diz: ``Eu quero passar para
vocês uma informação. Vocês provavelmente não acreditem e por não
acreditarem, precisam fazer o teatro da crença através das
representações. Ele (Deus) existe!''.

 

Jesus faz a mesma coisa, como Martin Luther King faz a mesma coisa. Como
no Brasil, Frei Tito fez a mesma coisa na época da ditadura. Todas às
vezes quando você de maneira mais simples diz: ``Só sou um pedaço do
Todo, agora o Todo existe''.

 

\textbf{R:}~Nesse sentido há a ideia concomitante da angústia em ser
hipócrita. As pessoas transformaram a hipocrisia em mais um pecado, que
gera muita angústia. A~pessoa diz: não, eu vou me comportar, porque eu
não sou assim.

 

Esse tipo de questão do sujeito comum, acaba gerando justamente essa
angústia: a pessoa não é ser um e único, e portanto não é, nesse
sentido, a imagem de Deus.

 

Essa ideia de você não ser à imagem de Deus, é o que acaba causando a
ideia da hipocrisia, a ideia de trair a si mesmo. A~pessoa está traindo
a si mesma no momento em que ela está se comportando de uma forma
teatralizada, que é eventualmente mais pertinente à situação que ela
está enfrentando.

 

Nesse sentido, como é, na sua opinião, que essa falta de percepção da
adaptação do ser ao mundo, seja ela um pensamento judaico, seja ela um
pensamento adaptativo darwiniano, atua na criação do mal"-estar, digamos
assim, do indivíduo na civilização?

 

\textbf{G:}~Bom, eu gostaria de deixar claro o seguinte: A minha
perspectiva é de que\textbf{}~nós fomos sim criados à imagem e
semelhança de Deus. Daí eu denominar o meu trabalho de Psicologia
Imagética.

 

A gente faz o tempo todo uma imagem, e um trabalho em cima da aparência.
Nós só somos aparência. Daí nós sermos um pedaço. Nós não somos pequenos
deuses como o totalismo religioso, político, ideológico, filosófico,
científico pretende. Neste sentido, nós não somos muito importantes.

 

Mas nós somos importantíssimos, exatamente na medida em que nós somos
essa imagem, em que nós podemos fazer essa representação. Quanto mais
nós tivermos condição de transformar o mundo, e transformar nossa
própria vida através das máscaras que nós usarmos, dos artifícios que a
gente vai fabricando, mais poderosos nós seremos. E~mais nós intervimos
na ordem das coisas. A~ideia é de que Deus precisa de nós. Da mesma
forma que nós precisamos de Deus.

 

Segundo a lenda, Deus criou o homem para não ficar solitário. Isto é
extraordinário. É~exatamente isto que explica o anseio pela procriação e
pela arte. Nós precisamos acontecer, é o \emph{happening.}

 

A hipocrisia é uma representação burlesca e é uma farsa empobrecida. Os
medíocres são hipócritas e os medíocres são perversos. Os medíocres são
dementes e são sádicos. A~doença é hipócrita. Quando o sujeito se
aproxima um pouquinho do seu papel na cena, ele vai se afastando da
hipocrisia e vai se afastando, vamos chamar assim, à falta talvez de um
melhor conceito, vai se afastando do Nada. Ele passa a acontecer.

 

\textbf{R:}~Quer dizer,\textbf{}~na medida em que o indivíduo participa
da cena com uma maior percepção, e maior empenho, ele recupera a sua
essência?

 

\textbf{G:} Exatamente, nesse momento ele se transforma em apóstolo.
Esse é o apostolado. Essa é a sacralidade. Essa é a transformação da
desimportância em importância. Quando penso na minha vida, fica nítido
que os momentos pontuais de significância foram aqueles momentos em que
a solenidade fica presente. Em que o banal foge. Nós permanentemente
estamos diante e no interior do grande espetáculo.

 

Lembra de certa maneira aquele filme ``O Mundo de Truman''. É~uma
alegoria medíocre, mas é disso que se trata.

 

\textbf{R:}~Na sua opinião, a religião é muito mais uma regra de atuação
da humanidade, do ser humano e da humanização, do que efetivamente algum
tipo de louvor a uma religiosidade?

 

\textbf{G:}~A religião, quando ela se institucionaliza, é na minha
opinião, um teatro\textbf{}~mal produzido. É~um teatro menor. Em matéria
de teatro, Shakespeare é bem melhor do que o Papa. Mas religião é um
teatrinho mambembe, principalmente para pessoas que se contentam com
pouco.

 

Mas aqueles que só se contentam com a religiosidade, aqueles que buscam
a maior, que buscam a poesia, que buscam realmente Bach, mas de verdade
Bach, esses estão o tempo todo ouvindo as trombetas. Aliás, este cenário
é o cenário que eu sempre valorizei na minha vida pessoal. Eu sempre
tive muita consciência da minha fragilidade e do pequenino que eu sou.
Absolutamente sempre tive consciência. E~quando eu não tive consciência,
os outros me lembravam disso.

 

Eu quero deixar claro, que ao mesmo tempo eu sempre tive muita
consciência de estar participando de um grande espetáculo; e aí, embora
como um coadjuvante simplesinho, eu sempre tentei dar o recado.

 

Eu sou capaz de estar conversando aqui com você, e me lembrar de alguns
momentos em que eu ouvia as trombetas tocando. E~esses momentos em que
eu ouvia as trombetas tocando, sempre foram os momentos que justificaram
minha vida. O~resto eu acho uma chatice, uma mesmice insuportável. Se eu
tivesse que viver sempre a mesmice do cotidiano e da rotina, eu não me
suportaria definitivamente.

 

\textbf{R:}~Então basicamente podemos identificar aí um dos indícios da
neurose?

 

\textbf{G:}~A maior parte das pessoas que me procuram não está
preocupada com\textbf{}~sexo. Isso é pretexto. Essa ideia de sexo, mesmo
essa ideia elaborada de Édipo, é uma bela fantasia do Freud com a mãe
dele, é ficção dele. Aliás, é uma belíssima história de amor das
dificuldades entre Freud e o pai dele, Jacó. Mas ela é só uma história,
como milhares que eu tenho ouvido no consultório.

 

Muitos têm obsessão com o pai e têm fixação com a mãe, outros têm com a
tia, outros têm com a prima. Isso é de uma desimportância absoluta.

 

As pessoas estão sempre, na realidade, com uma pergunta só. A~pessoa vai
fazer análise, na minha opinião, ou a pessoa vai para um oratório, ou a
pessoa vai para a igreja, ou a pessoa vai fazer um filho, ou a pessoa
vai para a guerra, sempre, na minha opinião, em busca de um sentido para
a sua vida.

 

Ela quer justificar o seu papel no grande espetáculo. Primeiro ela quer
saber qual é o seu papel. A~maior parte das pessoas está perdida. Tem
dificuldade, não conseguem ler um roteiro, mesmo porque a gente sabe que
a grande mistificação está sempre presente, através da mídia, das
intermediações. Não existe nada que sirva mais para desinformação do que
a mídia dos grandes interesses das corporações, da oligarquia.

 

\textbf{R:}~Mas dentro das suas palavras, a mídia é o roteiro
daquele\textbf{}~que não sabe o seu roteiro.

 

\textbf{G:}~É exatamente isto. A~mídia é o roteiro de quem não sabe o
seu roteiro. Só\textbf{}~que quem faz a mídia são os donos da mídia.
Porque os jornalistas, em boa parte, são pobres diabos. A~mídia serve a
instrumentalização midiática, do mesmo jeito que servem os sermões do
Papa, os discursos estúpidos e as orações vazias dos rabinos, as
pregações evangélicas na televisão, os anúncios das grandes corporações
televisivas, os grandes jornais.

 

Eu escrevi e atuei praticamente em todos os órgãos de informação do
Brasil. Aliás, eu acho muito curioso, porque todas as vezes no começo,
eu sou incensado. Eles abrem espaço, oferecem um tempo maior que se
poderia requerer. Sou considerado uma das pessoas mais importantes do
país em vários períodos da história e assim por diante. Mas basta que eu
comece a falar… Porque é óbvio que eu uso tudo isso como
trincheira, eu uso como nicho de informação. É~uma estratégia para
representar o meu papel.

 

Mas quando eu começo a representar o meu papel, as portas se fecham
imediatamente. Não foi para isso que nós chamamos você, nós queríamos
você para garoto"-propaganda. Mas chega um momento em que o engano mútuo
deixa de existir. Mas eu vou aproveitando, enquanto der, para dar uma ou
outra informação, eu aproveito. Isto começa a ser, de certa maneira,
rompido com o mundo virtual.

 

Não por acaso eu associo isto ao Norbert Wiener que escreveu God \&
Golem, Inc., e que foi na realidade o cérebro fundador da ideia binária,
do um dois, que atribui ao aprendizado com o avô dele, numa escolinha
judaica chamada \emph{cheder}, aonde o aprendizado se faz através do
\emph{pilpul,} que é o diálogo de duas pessoas.

 

Duas pessoas que sentam, conversam, discordam, concordam a tese,
antítese, a síntese. A~falta de síntese, a superação da síntese, a
conversa como finalidade em si mesma, sem preocupação nenhuma de
descoberta daquilo de que alguma maneira se interrogue, a narrativa de
começo, meio e fim no teatro. A~quem interessa, a quem serve a ordem?
Mas se existe alguma coisa que realmente empresta vida não é a ordem, é
a desordem. É~a desordem que é criativa, é a desordem que reformula.

 

\textbf{R}: Mas as pessoas entendem desordem como uma total falta de
limites. Na\textbf{}~verdade essa proposta de ensinar é anterior ao
Wiener, porque ela é descrição de Platão dos diálogos de Sócrates. Então
talvez a gente esteja usando dois tipos referenciais.

 

Mas, a desordem ela tem limites. No Banquete por exemplo, Sócrates,
quando escuta uma pergunta finalmente séria para ele, põe água no vinho
para diminuir a possibilidade de ficar embriagado. Na hora que eu estou
falando sério, não estou querendo me embriagar para dizer qualquer
coisa.

 

Então esta sua desordem, que eu compreendo, mas acredito que a maioria
não compreenda, o quanto ela tem limites, ou ela não tem limites?

 

\textbf{G:} Na verdade, quando eu falo em desordem, eu estou falando em
anarquia no\textbf{}~sentido filosófico. Que é a liberdade plena. Não
existe meia liberdade, como dizem os franceses, não existe demi"-vièrge
(semivirgem).

 

Ou você é livre, ou você é submisso. Aí voltamos à questão de sair do
Egito. Eu não sei se você se lembra, tem um determinado instante muito
importante, em que o povo aflito com as dificuldades do deserto ameaça a
sublevação contra Moisés e diz ``nós estávamos melhor no Egito, lá nós
tínhamos tudo em ordem, tínhamos casa. É~verdade que tinha um chicote do
verdugo, mas estava tudo em ordem.'' Essa ordem é o que explica todas as
tiranias.

 

A liberdade, ela tem sempre um senso e nisto vem o conceito de graça. Eu
tenho feito um trabalho com o Eduardo Sterblitch que eu estou chamando
de teatro psicológico do real, onde eu tento levar a psicologia
imagética para o palco com um grupo de humoristas do programa ``Pânico
na \versal{TV}''. É~mais ou menos a pantomima da Graça.

 

Na última vez, comentei que a minha mãe, num dos livros que escreveu,
tem um verso que eu acho muito bonito, que diz ``Mentira, você é uma
graça''. ``Aí ela usou, é óbvio, os dois sentidos: daquilo que faz rir,
e daquilo que também é santificado''. Não esquecendo que algumas figuras
das mais patéticas e ao mesmo tempo mais interessantes da cultura e da
religiosidade eram chamados ``loucos de Deus''.

 

E na outra ponta, Dom Quixote, que de certa maneira implementa a
desordem absoluta, quando junta a fantasia com a outra fantasia que é a
realidade.

 

\textbf{R}: Muito interessante essa colocação, porque existe diferença
entre João da Cruz e\textbf{}~Dom Quixote basicamente, é que o juízo da
realidade que Dom Quixote faz do mundo, não é compartilhado por todos os
seus pares. Assim como o juízo de realidade que João da Cruz faz do
mundo, também não é compartilhado por nenhum dos seus pares. A~diferença
é que o juízo de João da Cruz é feito em nome de Deus e de Dom Quixote é
feito em nome de si mesmo.

 

Aquele que é feito em nome de si mesmo é considerado louco. Aquele que
faz em nome de Deus é considerado profeta, devoto, apóstolo. De alguma
maneira quando o juízo é feito em nome de alguma coisa maior, todo mundo
consente.

 

Eu pergunto então, a loucura é na verdade o juízo de realidade vivido a
partir de uma centralidade humana tão exacerbada, que impede que os
outros compartilhem daquele juízo de realidade, e passam a desejar
portanto, expelir a pessoa do palco, da peça, da atuação?

 

\textbf{G:} Eu diria que a loucura é talvez o estado de espírito mais
trágico da condição\textbf{}~humana. Porque é a perda da
ego"-centralidade. O~sujeito deixa de ser ele mesmo, para ser tomado.
Tanto é que tem a ideia grega do ``daimon''.

 

Tem a ideia dos indígenas brasileiros que quando o sujeito enlouquece,
eles chamam de doença do susto. Eu falo disso no livro ``O Feitiço da
Amérika'', que é bem característico dessa ideia do sujeito possuído
por uma porção de realidades, mas que ele não possui.

 

Todos nós temos essas realidades, todos nós temos uma porção de pedaços.
Agora, precisamos aprender a conviver com esses pedaços, não com o
objetivo de inteireza, que é um ilusão. Aliás, essa ilusão costuma ser
um dos indicativos da própria da própria loucura. O~conceito ``Eu sou
Napoleão'', ``Eu sou Deus'', ``Eu sou todo poderoso'', ``Godplayer''.

 

Não existe uma estética aparentemente tão sedutora quanto a estética da
ordem. Quando você vê, por exemplo, um desfile militar, cinquenta mil
sujeitos todos batendo o pé, em uníssono, dá uma sensação ilusória de
fantasia.

 

Mas, quando você assiste um acrobata numa pirueta desordenada, aí sim
existe uma entrega ao Todo que não tem nada a ver com a loucura. Tem a
ver com a liberdade e com o mergulho nessa consciência do Eterno, é um
sair de si mesmo sendo você mesmo.

 

\textbf{R:}~Na medida em que, acredito, nenhum de nós dois percebe que
existe algum sentido em falarmos em cura psicanalítica, o que seria o
atendimento ou o sentido do atendimento? Seria permitir que a pessoa em
sofrimento possa enxergar quais são todos os seus pedaços, todos os seus
fragmentos e consiga aceitar os seus pedaços, os seus fragmentos,
justamente para participar desse teatro, desse roteiro, dessa cena na
qual ela está imersa?

 

\textbf{G:}~Eu vejo da seguinte maneira: psicanálise, psicoterapia,
psicologia,\textbf{}~sociologia, religião, ciência, medicina, são todos
instrumentos criados pela cultura para me tirar do desamparo e do
sofrimento humano. A~sociologia pretende dar uma informação de como é
que as coisas no universo do social acontecem. A~cultura toda tem esse
papel e a psicanálise é inserida dentro disso. Agora, a arrogância
científica e a arrogância da cultura são extremamente perigosas.

 

A psicanálise não cura nada, nem ninguém. A~sociologia não explica
absolutamente nada. A~medicina é esse fiasco terrível que nós sabemos.
Dizendo numa semana e desdizendo na outra.

 

Nós somos, e aí é para caracterizar bem a questão da fragmentação, nós
somos entes perdidos andando pelo deserto. Nós não chegamos ainda à
Terra da Promissão. Isso é mentira.

 

A mentira de que vai existir uma cura psicanalítica, por isso o
sujeito chega ao paraíso, chega ao êxtase. De que ele usando uma droga
chega ao paraíso. Que ele frequentando uma religião, que ele pagando
convênio médico ou indo para uma academia e fazendo seiscentas
abdominais, vai ficar invicto e não vai morrer. Tudo isso é o mundo
terrível das ilusões.

 

Nós ainda estamos naquela posição da caverna. Nós estamos ali vendo
sombras e imagens. Por isso que eu gosto da ideia da psicologia
imagética. O~que se faz num consultório é o encontro de duas pessoas. Aí
vai depender de uma série de circunstâncias. O~fundamental é que essas
duas pessoas se encontrem, que esse rapport se estabeleça num esforço de
graça e de alegria. É~um momento de iluminação, aquilo que eu chamei de
ouvir as trombetas.

 

Palmilhando a minha história e voltando àquela passagem, eu sou capaz
perfeitamente de fazer um relatório informando cada momento em que eu
ouvia a trombeta. E~cada momento, e esses são em número muito maiores,
em que eu vivi o pesadelo do banal.

 

É o momento que a gente pode chamar de zen, se a gente quiser usar uma
expressão japonesa, ``insight'', expressão americana; o momento em que
você se sente vivo e vibrando com a vida. Em que a vida penetra nesse
pedaço que você é, e através daí existe uma inteligência exponencial
extraordinária. Eu chamaria, talvez, isso de epifania carismática ou algo
parecido com isto. Você encontra isso na religião, ou na medicina ou na
psicanálise ou na psicoterapia. É~disso que se trata encontrar esse
momento e dizer que está valendo a pena viver.

 

Porque o medo da gente, e esse é o vestibular do desespero, é a
interrogação; se vale a pena.

 

Quando Deus ordena a Abrãao, quando ele tem cem anos, ``sai de você''
--- lech lechá, sai dessa pretensão de um você que não existe. Você não
é você, você é um relâmpago, você é uma passagem. Você é a travessia,
você é a páscoa, você é o pessach. Você é o ir e vir entre o antes e a
morte. Esse você é o menos importante, muito menos importante, pouco
importante. O~fundamental é esta consciência dessa continuidade.

 

\textbf{R:}~Primeiro uma constatação: isso quer dizer que na sua
opinião, seja\textbf{}~medicina, seja religião, seja academia de
ginástica, seja o partido político, a ideologia ou o que for,
basicamente todas essas propostas estão apenas sugerindo um
preenchimento daquilo que estaria vazio caso a gente simplesmente
nascesse para esperar a morte?

 

\textbf{G:}~É, mas é um preenchimento, que em si, voltando à medicina,
academias, e\textbf{}~assim por diante, o preenchimento só vai ter algum
significado se ele for um pedal, se ele for um instrumento. Ele não pode
ser um fim em si mesmo. Não adianta nada você esculpir um corpo, eu ouço
muito por aí: ele é bonito como um deus grego. Então estamos falando de
estatuária.

 

\textbf{R:}~Mas é exatamente isso que eu imagino que as pessoas dizem,
eu pelo\textbf{}~menos sempre ouvi assim, ele é tão bonito como uma
estátua de proporções heroicas.

 

\textbf{G:}~Exatamente, e que coisa pobre que é isso. E~o quanto é
interessante, exatamente o contrário. Quando eu falo sempre, e aqui eu
repeti várias vezes hoje, na questão da solenidade, do significado, de
ouvir a trombeta, em geral isso é muito sutil. É~num cruzar de olhares,
numa sensação de sentir o vento passando, uma palavra de esperança, uma
palavra de dor. É~sempre algo que fica muito perto do oculto e do
revelado.%\nota{esta entrevista está disponível na íntegra em~goo.gl/FhRcpw.} 

\fechafala 
