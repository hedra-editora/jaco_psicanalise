\chapterspecial{Posfácio}{Goldberg --- O Psicólogo da Poética}{Sandra Magalhães}
 

Em sua amplitude, Jacob Pinheiro Goldberg vem construindo uma teoria
brasileira da psicologia, na melhor tradição da originalidade que
captura no humano, a liberdade. Aproprio"-me das palavras de Gilberto
Felisberto Vasconcellos que, em crítica publicada na Folha de S\,Paulo,
sobre Cultura da Agressividade, a edição da tese de doutoramento em
psicologia de \versal{JPG}, o enquadra ``à maneira dos novos filósofos franceses,
que são os representantes do pós"-modernismo''.

Navegando à deriva, não por acaso acaba Crusoé e na ilha poética. Nesta,
é revelado por Marilia Librandi Rocha (tese de doutoramento em teoria
literária e literatura comparada, Parábola e Ponto de Fuga, a Poesia de
Jacob Pinheiro Goldberg, \versal{USP}) e, confirmado, na literatura o espaço,
como ``Uma poesia que se engana de endereço, do mesmo modo que esse eu
--- lírico diz não ser daqui''.

Psicologia em Curta"-metragem é outro turn over na difícil e
impressionante obra de Goldberg. Difícil enquanto escrita e falada
dispersa e exilada, em conferências, artigos, no Brasil e no exterior,
em um processo de turbilhão de um peregrino que, em torno de si mesmo,
visita o Outro e o mundo com profunda militância social que já se
incorporou à própria história de nossa cultura.

Impressionante, porque levanta polêmica, contesta, raciocina e faz a
apologia da desrazão, provocando a ira dos covardes de espírito,
suscitando emoção no humano. Paradoxalmente e como reflexo de suas sete
vidas, sua oratória seduz em promessas que reverberam o messiânico. E,
por isso, particularmente, convido o leitor a procurar a exposição
Psicologia do Sentenciado em que \versal{JPG}, como advogado na Procuradoria
Geral de São Paulo e na Faculdade de Direito da \versal{USP}, lança uma fórmula
legal que mobiliza, segundo a jurista Ana Sofia Schimidt de Oliveira,
uma leitura complexa e transformadora do apenado. Conteúdos que surgem
em conferencia -- na Faculdade de História da \versal{USP} defendendo a concessão
do titulo de doutor honoris"-causa, para Sobral Pinto; em debate
publicado no jornal O Primeiro de Janeiro, Porto, Portugal; e na Revista
da \versal{OABMG} com o Professor Doutor Piquet Carneiro (Harvard e Instituto
\versal{FHC}) sobre violência.

Cinéfilo por circunstâncias existenciais, exige do leitor que reescreva,
na conformidade de seu tamanho, uma concepção de ser a partir do
personagem.

Definitivamente, mais um livro para adulto e termino como iniciei,
pedindo o aval de Oscar D'Ambrosio, que no Jornal da Tarde escreveu a
respeito de Ritual de Clivagem de Goldberg: ``A fragmentada exposição de
uma experiência individual compõe o painel de uma sociedade que se
busca, estilhaçando"-se em cada passo rumo à redenção de uma psique cada
vez mais compreendida''.

Para quem carrega uma dolorida Juiz de Fora no desterro, país em que
nasceu.

Regina Przybycien, professora da Universidade do Paraná, analisando
Magya Wygnania, (tradução polonesa de Henryk Siwierski, professor das
Universidades de Cracóvia e Brasília, da antologia Mágica do Exílio, de
\versal{JPG}), adianta que se trata de um outsider, comparando"-o a Tadeusz
Rozewicz e a Carlos Drummond de Andrade. \versal{CDA} que escreveu antes de
morrer sobre \versal{JPG}: ``… Refletindo uma consciência crítica e uma
sensibilidade intensa diante do espetáculo contraditório e brutal do
mundo de hoje, é uma reação vital de quem acredita nos valores humanos e
tenta preservá"-los contra a barbárie crescente… \redondo{[…]}
alcance compreensão e ressonância a sua Mensagem…''

É natural, assim, que o cinema teria de ser mais uma fronteira da
geografia de Goldberg. Certamente a penúltima deste intelectual que leva
a inteligência e o ritmo ao limite. Falta pouco para Quixote desembarcar
em Passárgada.

 

\chapter*{\mbox{}} 

\emph{Jacob Pinheiro Goldberg}~é Doutor em psicologia, psicólogo, e
professor convidado das seguintes instituições:
University College London Medical -- Universidade Eotvos
Liorand(Hungria) -- Universytet Jagiellonski e Universytet Warszawski
(Polônia); Middlesex University (Inglaterra); Hebrew University of
Jerusalém; \versal{USP} -- \versal{PUC}/\versal{SP} -- \versal{PUCC} -- Universidade de Brasília -- \versal{UNESP} --
Mackenzie, Aspirus Wausau Hospital, Wisconsin (E.U.A.).É também
advogado, assistente"-social e escritor.

 

\section{Obras publicadas}

\begin{itemize}
\itemsep1pt\parskip0pt\parsep0pt
\item
  Imagética Psicológica. 2015
\item
  Juiz de fora, dentro. 2015
\item
  Imaginetica Ed Saraiva 2015
\item
  Não calo, Falo. 2015
\item
  Cachoeira, á transbordar. Ed. Saraiva 2015
\item
  Do sonho, solo. Ed. Saraiva 2015
\item
  Bom dia, falta. Ed. Saraiva 2015
\item
  De la locomotora a Leña hasta El Picadelly Circus. Ed. Saraiva 2015
\item
  Eco e reverberação \versal{ED}. Saraiva 2015
\item
  O próximo do mundo. Ed Saraiva 2015
\item
  Na imprensa em Juiz de fora
\item
  O percurso Ed. Saraiva 2015
\item
  O psicologo e o jornalista Ed. Saraiva 2015
\item
  Na cena Ed Saraiva 2015
\item
  Psicanalise da morte Ed saraiva 2015
\item
  Psicanalise e metanoia Ed saraiva 2015
\item
  \versal{EU} \versal{QUE} \versal{NÃO}. Ed Saraiva 2015
\item
  ``Psicanalise da morte'' Ed. Saraiva 2015
\item
  Poesia --- Ed. Saraiva 2014
\item
  Laboratorio de Literatura Monteiro Lobato --- Ed Saraiva 2014
\item
  Entre o sol que se tenta a sombra da ultima neblina --- Ed Saraiva
  2014
\item
  Palavra e imagem na transformação --- Ed. Saraiva 2014
\item
  Psicanalise e Metanoia Ed. Saraiva 2014
\item
  Na Cena --- Ed. Saraiva --- 2014
\item
  Golem, Anverso --- Ed. Saraiva --- 2014
\item
  Stefan Zweig --- Ed. Saraiva --- 2014
\item
  ``Psicologia ao acaso --- Ideias para um dia melhor'' --- Ed. Amazon-
\item
  ``O Percurso'' --- Ed. Saraiva --- 2014
\item
  `` Eis que todos saibam'' --- \versal{ED} Saraiva --- 2014.
\item
  ``O feitiço da Amerika'' --- 2013 Ed.Amazon -Virtual-
\item
  Goldberg prefáciou ``A mocinha do Mercado Central'' --- Prêmio Jabuti
  de Literatura, 2012.
\item
  ``Sentido e Existência'' --- Com palestras para as Universidades
  Stanford, Crocovia, Lublin e Brasília --- Colombo Studio, 2012.
\item
  ``O Direito no Divã'' --- Organização Flávio Goldberg, Editora
  Saraiva, 2011.
\item
  Monólogo a Dois. Google Books, 2010.
\item
  Mocinhos e Bandidos --- Prefácio. Google Books, 2010.
\item
  Ética e Tecnologia. Google Books, 2010.
\item
  Psicologia do Sentenciado. Google Books, 2010.
\item
  Ritmo Esquerdo. Google Books, 2010.
\item
  Psicologia da Agressividade. Google Books, 2010.
\item
  Segunda Madrugada. Google Books, 2010.
\item
  Perspectivas da Literatura segundo Goldberg. Google Books, 2010.
\item
  A Morte de Stefan e Elisabeth Zweig. Google Books, 2010.
\item
  Memórias do Abismo. Google Books, 2010.
\item
  Tempo Exilado. Google Books, 2010.
\item
  O Direito e a Ordem Jurídica nos Processos do Desenvolvimento. Google
  Books, 2010.
\item
  Maneco Nheco Nheco. Google Books, 2010.
\item
  Cantata para o Brasil. Google Books, 2010.
\item
  Rua Halfeld, Ostroviec. Google Books, 2010.
\item
  Psicoterapia e Psicologia.~Google Books, 2010.
\item
  Teoria Social da Comunicação. Google Books, 2010.
\item
  A Ógea e a Calhandra. Google Books, 2010.
\item
  Violência Urbana. Google Books, 2010.
\item
  Maneco. Google Books, 2010.
\item
  Judaismos: Ético e não"-étnico. Google Books, 2010.
\item
  O Percurso. Google Books, 2010.
\item
  Freud e o Ocultismo. Google Books, 2010.
\item
  A Tautology on Violence: from the viewpoint of Justice and
  Psichology.~Google Books, 2010.
\item
  Cantata para o Brasil --- Ensaio. Vanessa Leite Barreto Quintino.
  Google Books, 2010.
\item
  O Feitiço da Amérika. Google Books, 2010.
\item
  História que a Cigana (nua) me contou. Google Books, 2010.
\item
  Penúltima Estação. Google Books, 2010.
\item
  O dia em que Deus viajou. Google Books, 2010.
\item
  Poemas Vida --- Antologia de Jacob Pinheiro Goldberg. Google Books,
  2010.
\item
  Atuação Social e Científica de Jacob Pinheiro Goldberg. Google Books,
  2010.
\item
  Antepenúltima Estação. Google Books, 2010.
\item
  Um Romance de Vida. Google Books, 2010.
\item
  Serviço Social no Exército Brasileiro. Google Books, 2010.
\item
  A Poesia de Fanny Goldberg: Uma mulher, muitas vozes. Google Books,
  2010.
\item
  Parábola e Ponto de Fuga: A Poesia de Jacob Pinheiro Goldberg --- Vol.
  1. Marilia Librandi Rocha. Google Books, 2010.
\item
  Psicologia no campo da Medicina. Google Books, 2010.
\item
  Parábola e Ponto de Fuga: A Poesia de Jacob Pinheiro Goldberg. Google
  Books, 2010.
\item
  A discriminacão racial e a lei brasileira. Google Books, 2010.
\item
  Cidade dos Sinos. Google Books, 2010.
\item
  Comunicação e Cultura de Massa. Google Books, 2010.
\item
  Co"-autor de Psiquiatria Forense e Cultura. Vetor Editora, 2009
\item
  Psicologia em Curta"-Metragem. São Paulo: Novo Conceito, 2008
\item
  Poemas"-Vida --- Antologia organizada por Marília Librandí Rocha, 2008
\item
  Prefácio de Mocinhos e Bandidos --- Controle do Conteúdo Televisivo e
  Outros Temas, 2005.
\item
  Rua Halfeld, Ostroviec --- Open Press, 2005
\item
  Cultura da Agressividade. São Paulo: Landy, 2004
\item
  A Mágica do Exílio. São Paulo: Landy, 2003.
\item
  Prefácio de ``História da Morte no Ocidente'', de Philippe Ariès, 2003
\item
  Prefácio de ``O Peso de uma Aposta'', de Sérgio Bustamante, 2003
\item
  Monólogo a Dois, 2002
\item
  Comentário em ``As chaves da Gotte des Fées'', da Profa. Dra. Ria
  Lemaire, 2001
\item
  Colaborador em ``Retroviroses Humanas --- \versal{HIV}/\versal{AIDS}'' de Roberto
  Focaccia, 1999
\item
  Psicologia de Imagem --- Faculdade de Psicologia da P.U.C\,--- 11 de
  maio 1999
\item
  * Prefácio de ``Shakespeare não serve de Álibi'' de Lucinio Rios.
  1998.
\item
  Judaísmos: Ético e não"-étnico --- Capital Sefarad Editorial, 1997
\item
  Don't let me die --- publicado pela ``Women S.O.S.'' --- Oakland,
  E.U.A\,\item
  A Ógea e a Calhandra --- Capital Sefarad Editorial --- 1997.
\item
  A Clave da Morte --- Editora Maltese, 1994.
\item
  O Feitiço da Amerika --- Edição Popiatã, 1991
\item
  Ritual de Clivagem --- Ed. Massao Ohno Editor, 1989
\item
  Poesia publicada em ``International Poetry'' --- Universidade do
  Colorado, \versal{EUA}, 1986
\item
  Carta publicada em ``International Poetry'' de Teresinka Pereira 1985
\item
  Psicologia da Agressividade --- Ed. \versal{ICC}, 1983
\item
  Citação em ``Geração Abandonada'', de Luiz Fernando Emediato, 1982
\item
  Citação em ``Crise Social e Delinquência'', de James Tubenchlak, 1981
\item
  Psicologia e Psicoterapia --- Ed. Símbolo, 1979
\item
  Maneco --- Ed. Nova América
\item
  Historic Invention and Psychological Understanding of Jesus, 1978
\item
  Psicologia e Reflexões do Inconscinte --- Ed. \versal{OINAB}, 1978, 1a edição
\item
  Cantata para o Brasil --- Ed. \versal{OINAB},1978,2 a edição
\item
  Clave da Morte, Google Books, 1978
\item
  Origem do pensamento freudiano, Planeta, 1977
\item
  Freud and the Spirit Manifestation, 1977
\item
  Indoamerika --- Ed. Unidas Ltda. 1976
\item
  Cidade dos Sinos --- Ed. Clássico Cientifica, 1975
\item
  Perspectivas da Literatura segundo Goldberg --- Ely Vieitez Lanes,
  1975
\item
  O Dia em que Deus Viajou --- Ed. Clássico Cientifica, 1974
\item
  Comunicação e Cultura de Massa --- Ed. Cultural, 1972, 2 edição
\item
  Memórias do Abismo --- Ed. Cultural, 1972
\item
  Monólogo do Medo --- Ed. Cultural, 1972.
\item
  Comunicação e Cultura de Massa --- Ed. Cultural, 1972, 2 edição
\item
  Segunda Madrugada --- Ed. Cultural, 1971
\item
  Teoria Social da Comunicação --- Ed. Cultural, 1969
\item
  Tempo Exilado, Ed Cultural, 1968
\item
  Ética e Tecnologia --- Ed. Fulgor, 1968. Tempo Exilado --- Ed.
  Cultural, 1968
\item
  A Discriminação Racial e a Lei Brasileira --- Ed. Luanda.1966
\item
  História que a Cigana (nua) me contou, 1960
\item
  Ritmo Esquerdo --- Ed. Rio, 1954
\end{itemize}

 

\emph{Renato Bulcão} é professor de filosofia, formado pela \versal{USP} com
mestrado em Comunicação pela mesma universidade. Leciona filosofia na
\versal{UNIP} e arte audio"-visual em diversas instituições. Pesquisa o ensino a
distancia desde 1995, e publicou artigos em livros especializados.
Atualmente termina seu doutorado em Arte, Educação e História da Cultura
na Universidade Mackenzie.
